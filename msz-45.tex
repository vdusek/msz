% VUT FIT MITAI
% MSZ 2021/2022
% Author: Vladimir Dusek
% Login: xdusek27

%%%%%%%%%%%%%%%%%%%%%%%%%%%%%%%%%%%%%%%%%%%%%%%%%%%%%%%%%%%%%%%%%%%%%%%%%%%%%%%%

\chapter{Hledání minimální kostry obyčejného grafu (pojmy, stromy a kostry, Kruskalův algoritmus, Primův algoritmus).}

\section{Metadata}

\begin{itemize}
    \item Předmět: Grafové algoritmy (GAL)
    \item Přednáška:
    \begin{itemize}
        \item 5) Stromy, minimální kostry, Jarníkův a Borůvkův algoritmus.
        \item 6) Růst minimální kostry, algoritmy Kruskala a Prima.
    \end{itemize}
    \item Záznam:
    \begin{itemize}
        \item 2020-10-22
        \item 2020-10-29
    \end{itemize}
\end{itemize}

\section{Úvod a kontext}

\paragraph*{Orientovaný graf} Orientovaný graf je dvojice $G = (V, E)$, kde $V$ je konečná množina uzlů a $E \subseteq V \times V$ je množina hran.

\paragraph*{Neorientovaný graf} Neorientovaný graf je dvojice $G = (V, E)$, kde $V$ je konečná množina uzlů a $E \subseteq {V \choose 2}$ je množina hran. (Hrana je tedy dvouprvková množina, avšak běžně se držíme stejného značení jako u orientovaných grafů a používáme dvojici.)

\paragraph*{Ohodnocený graf} Ohodnocený graf je takový graf, jehož každá hrana má přiřazenou nějakou hodnotu, typicky definovanou pomocí váhové funkce $w : E \mapsto \mathbb{R}$.

\paragraph*{Podgraf} Graf $G' = (V', E')$ je podgraf grafu $G = (V, E)$ jestliže $V' \subseteq V$ a $E' \subseteq E$.

\paragraph*{Sled} Posloupnost uzlů $\langle v_0, v_1, \dots, v_k \rangle$, kde $(v_{i-1}, v_i) \in E$ pro $i = 1, \dots, k$ se nazývá sled délky $k$ z $v_0$ do $v_k$.

\paragraph*{Uzavřený sled} Sled $\langle v_0, v_1, \dots, v_k \rangle$ se nazývá uzavřený, pokud existuje hrana $(v_0, v_k)$.

\paragraph*{Dosažitelnost} Pokud existuje sled $s$ z uzlu $u$ do uzlu $v$, říkáme, že $v$ je dosažitelný z $u$ sledem $s$, značeno $u \xRightarrow{\text{s}} v$.

\paragraph*{Tah} Tah je sled ve kterém se neopakují hrany.

\paragraph*{Cesta} Cesta je sled ve kterém se neopakují uzly.

\paragraph*{Souvislý graf} Neorientovaný graf se nazývá souvislý, pokud mezi libovolnými dvěma uzly existuje cesta.

\paragraph*{Kružnice} Uzavřená cesta se nazývá kružnice.

\paragraph*{Cyklus} Orientovaná kružnice se nazývá cyklus (první a poslední uzel je shodný).

% todo
% \paragraph*{Prostý graf} Orientovaný graf se nazývá prostý, pokud...

\paragraph*{Acyklický graf} Acyklický graf je graf bez cyklů.

\paragraph*{Strom} Graf, který je souvislý a neobsahuje žádnou kružnici, se nazývá strom.

\paragraph*{Kostra} Strom, který je podgraf souvislého grafu na množině všech jeho vrcholů, se nazývá kostra.

\paragraph*{Minimální kostra} Nechť $G = (V, E)$ je souvislý neorientovaný graf s váhovou funkcí $w : E \mapsto \mathbb{R}$. Minimální kostra je souvislý, acyklický graf $G' = (V, E')$, kde $E' \subseteq E$ a $$w(E') = \sum_{(u,v) \in T} w(u, v)$$ je minimální.

\section{Generický algoritmus}

\todo{todo}

\section{Kruskalův algoritmus}

\todo{todo}

\section{Primův-Janíkův algoritmus}

\todo{todo}
