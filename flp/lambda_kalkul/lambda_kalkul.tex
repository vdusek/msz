% VUT FIT MITAI
% MSZ 2021/2022
% Author: Vladimir Dusek
% Login: xdusek27

%%%%%%%%%%%%%%%%%%%%%%%%%%%%%%%%%%%%%%%%%%%%%%%%%%%%%%%%%%%%%%%%%%%%%%%%%%%%%%%%

% Path to figures
\graphicspath{{flp/lambda_kalkul/figures}}

%%%%%%%%%%%%%%%%%%%%%%%%%%%%%%%%%%%%%%%%%%%%%%%%%%%%%%%%%%%%%%%%%%%%%%%%%%%%%%%%

\chapter{FLP -- Lambda kalkul (definice všech pojmů, operací, \dots).}

%%%%%%%%%%%%%%%%%%%%%%%%%%%%%%%%%%%%%%%%%%%%%%%%%%%%%%%%%%%%%%%%%%%%%%%%%%%%%%%%

\section{Zdroje}

\begin{compactitem}
    \item \path{FLP-FP.pdf}
    \item \path{flp_all_2018.pdf}
    \item \path{FLP_2020-02-03.mp4}
    \item \path{FLP_2020-02-10.mp4}
    \item \url{https://petrzemek.net/publications}
\end{compactitem}

%%%%%%%%%%%%%%%%%%%%%%%%%%%%%%%%%%%%%%%%%%%%%%%%%%%%%%%%%%%%%%%%%%%%%%%%%%%%%%%%

\section{Úvod a kontext}

\begin{compactitem}
    \item Lambda kalkul je formální systém (jazyk) a výpočetní model používaný v teoretické informace a matematice pro studium funkcí a rekurze.

    \item Lambda kalkul je teoretickým základem funkcionálního programování a příslušných programovacích jazyků (Lisp, Haskell).

    \item Lambda kalkul je výpočetně ekvivalentní turingovým strojům (a parciálně rekurzivním funkcím) (Church-Turingova teze) -- Jakýkoliv program může být vyjádřen v Lambda kalkulu.

    \item Funkce má ve funkcionálním programování (a tedy i v lambda kalkulu) význam matematický, tj. jedná se o čísté funkce (\textit{pure functions}). Výsledek závisí pouze na vstupních parametrech, funkce nemá žádný interní stav.

\end{compactitem}

%%%%%%%%%%%%%%%%%%%%%%%%%%%%%%%%%%%%%%%%%%%%%%%%%%%%%%%%%%%%%%%%%%%%%%%%%%%%%%%%

\section{Základy}

\begin{compactitem}
    \item Intuitivně, identita vs rovnost.
\end{compactitem}

\paragraph*{Proměnná} Proměnná jako z jiných programovacích jazyků.

\paragraph*{Abstrakce} Definice funkce.

\begin{equation}
    \begin{split}
        \lk x \lkdot x+2 &~~~~~ \text{odpovídá} ~~~~~ f(x) = x+2 \\
        \lk x \lkdot \lk y \lkdot x+y &~~~~~ \text{odpovídá} ~~~~~ f(x) = x+y \\
        \lk x ~ y \lkdot x-y &~~~~~ \text{odpovídá} ~~~~~ f(x) = x-y
    \end{split}
\end{equation}

\paragraph*{Aplikace} Volání funkce.

\begin{equation}
    \begin{split}
        ( \lk x \lkdot x+2) ~ 5 &~~~~~ \text{odpovídá} ~~~~~ f(5) \\
        ( \lk x ~ y \lkdot x-y) ~ 10 ~ 5 &~~~~~ \text{odpovídá} ~~~~~ f(10, 5)
    \end{split}
\end{equation}

\paragraph*{Výraz} Jako výraz, resp. $\lk$-výraz, se označuje každý z výše uvedených třech prvků (proměnná, abstrakce, aplikace). Typicky se značí velkými písmeny.

\paragraph*{Volná a vázaná proměnná} Rozlišujeme volnou a vázanou proměnnou. Vázanná proměnná je proměnná, která je parametr funkce. Ostatní proměnné jsou volné.

\begin{equation}
    \begin{split}
        \lk vazana \lkdot vazana + volna
    \end{split}
\end{equation}

\paragraph*{Zjednodušování zápisů}

\begin{equation}
    \begin{split}
        (A ~ B) &~~~~~ = ~~~~~ A ~ B \\
        (\lk E \lkdot (((A) ~ B) ~ C))) &~~~~~ = ~~~~~ \lk E \lkdot A ~ B ~ C \\
        (\lk A (\lk B (\lk C \lkdot x ))) &~~~~~ = ~~~~~ \lk A ~ B ~ C \lkdot x
    \end{split}
\end{equation}

\paragraph*{Pojmenování výrazů}

$$ \lklet K = \lk x \lkdot x+2 ~~~~~ \text{poté K můžeme používat v jiných výrazech} $$

%%%%%%%%%%%%%%%%%%%%%%%%%%%%%%%%%%%%%%%%%%%%%%%%%%%%%%%%%%%%%%%%%%%%%%%%%%%%%%%%

\section{Konverze (redukce)}

\begin{compactitem}
    \item Upravování výrazů na jiné, často jednodušší, resp. jejich konverze (redukování).

    \item Konverze zapisujeme šipkou.
\end{compactitem}

\subsection{$\alpha$-konverze [alfa]}

\begin{compactitem}
    \item Přejmenování proměnných

    \item Provádění má formu substituce: $[W/V]$ znamená, že $V$ je nahrazeno za $W$.

    \item Lze uplatnit pouze na vázané proměnné.

    \item Nesmí se změnit význam výrazu -- vázaná proměnná se nesmí stát volnou a volná proměnná se nesmí stát vázanou.

    \item Příklad správného použití (substituce se bude muset dále vyhodnotit i ve výrazu $E$):
    $$
        \lk V \lkdot E \rightarrow_{\alpha} \lk W \lkdot E ~ [W/V]
    $$

    \item Příklad chybného použití ($y$ se z volné stalo vázanou):
    $$
        \lk x \lkdot x ~ y \rightarrow_{\alpha} \lk y \lkdot y ~ y
    $$
\end{compactitem}

\subsection{$\beta$-konverze [beta]}

\begin{compactitem}
    \item Aplikace funkce na argument.

    \item Provedení má opět formu substituce.

    \item Příklad správného použití (substituce se bude muset dále vyhodnotit i ve výrazu $E$):
    $$
        ( \lk V \lkdot E ) M \rightarrow_{\beta} \lk M \lkdot E ~ [M/V]
    $$

    \item Příklad chybného použití ($y$ se z volné stalo vázanou):
    $$
        ( \lk x ~ y \lkdot x ~ y ) ( x ~ y ) \rightarrow_{\beta} \lk y \lkdot (x ~ y) ~ y
    $$
\end{compactitem}

\subsection{$\eta$-konverze [eta]}

\begin{compactitem}
    \item Vyjadřuje ekvivalenci výrazů a jejich převoditelnost.

    \item Příklad správného použití (podmínka platnosti: $V$ není volné v $E$):
    $$ \lk V \lkdot E ~ V \rightarrow_{\eta} E $$

    \item Příklad chybného použití ($x$ je volné):
    $$ \lk x \lkdot (x ~ y) ~ x \rightarrow_{\eta} (x ~ y) $$

\end{compactitem}

%%%%%%%%%%%%%%%%%%%%%%%%%%%%%%%%%%%%%%%%%%%%%%%%%%%%%%%%%%%%%%%%%%%%%%%%%%%%%%%%

\section{Rekurze}

\begin{compactitem}
    \item Využívá tzv. pevného bodu. V matematice jako pevný bod označujeme bod, který se v daném zobrazení zobrazí sám na sebe.
\end{compactitem}

\todo{todo}
