% VUT FIT MITAI
% MSZ 2021/2022
% Author: Vladimir Dusek
% Login: xdusek27

%%%%%%%%%%%%%%%%%%%%%%%%%%%%%%%%%%%%%%%%%%%%%%%%%%%%%%%%%%%%%%%%%%%%%%%%%%%%%%%%

\chapter{Hledání nejkratších cest ze zdrojového uzlu do všech ostatních uzlů grafu (Bellman-Fordův algoritmus, Dijkstrův algoritmus).}

%%%%%%%%%%%%%%%%%%%%%%%%%%%%%%%%%%%%%%%%%%%%%%%%%%%%%%%%%%%%%%%%%%%%%%%%%%%%%%%%

\section{Metadata}

\begin{itemize}
    \item Předmět: Grafové algoritmy (GAL)
    \item Přednáška:
    \begin{itemize}
        \item 7) Nejkratší cesty z jednoho vrcholu, Bellman-Fordův algoritmus, nejkratší cesta z jednoho vrcholu v orientovaných acyklických grafech.
        \item 8) Dijkstrův algoritmus. Nejkratší cesty ze všech vrcholů.
    \end{itemize}
    \item Záznam:
    \begin{itemize}
        \item 2020-11-05
    \end{itemize}
\end{itemize}

%%%%%%%%%%%%%%%%%%%%%%%%%%%%%%%%%%%%%%%%%%%%%%%%%%%%%%%%%%%%%%%%%%%%%%%%%%%%%%%%

\section{Úvod a kontext}

\textit{Viz. \uv{Úvod a kontext} v předchozí otázce.}

\paragraph*{Cena cesty} Nechť $G = (V, E)$ je ohodnocený graf s váhovou funkcí $w: E \mapsto \mathbb{R}$. Cena cesty $p = \langle v_o, v_1, \dots, v_k \rangle$ je suma $$
w(p) = \sum_{i=0}^k w(v_i, v_{i+1})
$$.

\paragraph*{Cena nejkratší cesty} Cena nejkratší cesty z $u$ do $v$ je $$
\delta(u, v) = \begin{cases}
    min( \{ w(p) : u \xRightarrow{\text{p}} v \} ) \\
    \infty ~ \text{pokud cesta neexistuje}
\end{cases}
$$.

\paragraph*{Nejkratší cesta} Nejkratší cesta z $u$ do $v$ je pak libovolná cesta $p$ taková, že $w(p) = \delta(u, v)$.

\paragraph*{Cena cesty se záporným cyklem} Pokud na cestě z $u$ do $v$ existuje záporný cyklus (cyklus jehož celková cena je záporná), pak $\delta(u, v) = - \infty$.

\paragraph*{Záporné ohodnocení hran} Pokud na cestě z $u$ do $v$ neexistuje záporný cyklus, tak algoritmy pracují dobře i se záporným ohodnocením hran.

\paragraph*{Reprezentace cesty} Cestu reprezentujeme pomocí pole předchůdců $\pi$.

\paragraph*{Hledání nejkratších cest ze všech uzlů do jednoho} Tento problém lze řešit stejnými algoritmy. Graf se transponuje, provede se algoritmus pro problém \uv{hledání nejkratších cest ze jednoho uzlu do všech ostatních uzlů} a poté se transponuje zpět.

%%%%%%%%%%%%%%%%%%%%%%%%%%%%%%%%%%%%%%%%%%%%%%%%%%%%%%%%%%%%%%%%%%%%%%%%%%%%%%%%

\section{Pomocné funkce}



%%%%%%%%%%%%%%%%%%%%%%%%%%%%%%%%%%%%%%%%%%%%%%%%%%%%%%%%%%%%%%%%%%%%%%%%%%%%%%%%

\section{Bellman-Fordův algoritmus}

Slouží pro řešení v obecných grafech (mohou obsahovat cykly a záporné hrany).

%%%%%%%%%%%%%%%%%%%%%%%%%%%%%%%%%%%%%%%%%%%%%%%%%%%%%%%%%%%%%%%%%%%%%%%%%%%%%%%%

\section{Dijkstrův algoritmus}

Slouží pro řešení v acyklických grafech bez záporných hran.
