% VUT FIT MITAI
% MSZ 2021/2022
% Author: Vladimir Dusek
% Login: xdusek27

%%%%%%%%%%%%%%%%%%%%%%%%%%%%%%%%%%%%%%%%%%%%%%%%%%%%%%%%%%%%%%%%%%%%%%%%%%%%%%%%

% Path to figures
\graphicspath{{msp/regresni_analyza/figures}}

%%%%%%%%%%%%%%%%%%%%%%%%%%%%%%%%%%%%%%%%%%%%%%%%%%%%%%%%%%%%%%%%%%%%%%%%%%%%%%%%

\chapter{MSP~--~Regresní analýza.}

%%%%%%%%%%%%%%%%%%%%%%%%%%%%%%%%%%%%%%%%%%%%%%%%%%%%%%%%%%%%%%%%%%%%%%%%%%%%%%%%

\section{Zdroje}

\begin{compactitem}
    \item \path{todo}
\end{compactitem}

%%%%%%%%%%%%%%%%%%%%%%%%%%%%%%%%%%%%%%%%%%%%%%%%%%%%%%%%%%%%%%%%%%%%%%%%%%%%%%%%

\section{Úvod a kontext}

\begin{compactitem}
    \item \textbf{Korelační analýza} se zabývá vzájemnými (většinou lineárními) závislostmi, kdy se klade důraz především na intenzitu (sílu) vzájemného vztahu než na zkoumání veličin ve směru příčina -- následek.

    \item \textbf{Regresní analýza} se zabývá jednostrannými závislostmi. Jedná se o situaci, kdy proti sobě stojí vysvětlující (nezávislá) proměnná v úloze příčin a vysvětlovaná (závislá) proměnná v úloze následků (hledání závislostí mezi atributy). V podstatě jde o aproximaci souboru dat vhodnou funkcí (tzv. regresní funkce). \begin{compactitem}
        \item Na začátku regresní analýzy je třeba odhadnout typ funkce. K tomu slouží explorativní analýza, která se používá ke zjištění, jak cílový atribut závisí na ostatních atributech (na kterých a jak).
        \item Poté je třeba určit parametry regresní funkce, například pomocí metody nejmenších čtverců.
        \item V závěru je třeba model verifikovat, zda funguje i na datech, na kterých nebyl přímo trénován.
    \end{compactitem}

    \item Rozlišujeme různé typy: \begin{compactitem}
        \item Jednoduchá lineární regrese -- Cílový atribut závisí na jednom dalším atributu lineárně.
        \item Vícenásobná lineární regrese -- Cílový atribut závisí na několika dalších atributech lineárně.
        \item Nelineární regrese -- Cílový atribut závisí na dalších atributech nelineárně.
    \end{compactitem}

\end{compactitem}

%%%%%%%%%%%%%%%%%%%%%%%%%%%%%%%%%%%%%%%%%%%%%%%%%%%%%%%%%%%%%%%%%%%%%%%%%%%%%%%%

\section{Polynomiální regrese}

\begin{compactitem}
    \item Polynomiální regrese představuje proložení (aproximaci) zadaných hodnot polynomem.

    \item Postup: \begin{compactitem}
        \item Mějme datový soubor $Y$ reprezentovaný uspořádanou n-ticí: $$Y = (y_1, y_2, \ldots, y_n)$$
        \item cílem je najít takový polynom k-tého stupně:
        $$P_k(x) = p_0 + p_1 x + \ldots + p_k x^k$$
        \item pro který platí
        $$y_i = P_k(x_i) + e_i$$
        pro $i \in 1 \ldots n$, kde $e_i$ je odchylka (nebo také chyba). Koeficienty $p_0, p_1, \ldots, p_k$ jsou přitom voleny tak, aby součet druhých mocnin odchylek, resp. suma $$\sum_{i=1}^n e_i^2$$, byla co nejmenší.
    \end{compactitem}
\end{compactitem}

\subsection{Metoda nejmenších čtverců}

\todo{todo}

\subsection{Střední kvadratická chyba}

\begin{compactitem}
    \item Jedna z chybových metrik je tzv. střední kvadratická chyba (MSE, \textit{Mean Squared Error}).

    \item Mějme trénovací datový soubor $(X, Y)$, kde $X = (x_1, x_2, \ldots, x_n)$ jsou hodnoty ovlivňující proměnné a $Y = (y_1, y_2, \ldots, y_n)$ jsou hodnoty cílové proměnné, a regresní funkci $f$, která aproximuje datovou sadu $(X, Y)$.

    \item Výpočet chyby MSE regresní funkce $f$ na datovém souboru $(X, Y)$:
    $$ \text{MSE} = \frac{1}{n} \sum_{i=1}^n (y_i - f(x_i))^2 $$
\end{compactitem}

%%%%%%%%%%%%%%%%%%%%%%%%%%%%%%%%%%%%%%%%%%%%%%%%%%%%%%%%%%%%%%%%%%%%%%%%%%%%%%%%

%%%%%%%%%%%%%%%%%%%%%%%%%%%%%%%%%%%%%%%%%%%%%%%%%%%%%%%%%%%%%%%%%%%%%%%%%%%%%%%%

\section{Příklad}
