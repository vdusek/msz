% VUT FIT MITAI
% MSZ 2021/2022
% Author: Vladimir Dusek
% Login: xdusek27

%%%%%%%%%%%%%%%%%%%%%%%%%%%%%%%%%%%%%%%%%%%%%%%%%%%%%%%%%%%%%%%%%%%%%%%%%%%%%%%%

% Path to figures
\graphicspath{{msp/nahodna_promenna/figures}}

%%%%%%%%%%%%%%%%%%%%%%%%%%%%%%%%%%%%%%%%%%%%%%%%%%%%%%%%%%%%%%%%%%%%%%%%%%%%%%%%

\chapter{MSP~--~Náhodná proměnná, typy náhodné proměnná, funkční a číselné charakteristiky, významná rozdělení pravděpodobnosti.}

%%%%%%%%%%%%%%%%%%%%%%%%%%%%%%%%%%%%%%%%%%%%%%%%%%%%%%%%%%%%%%%%%%%%%%%%%%%%%%%%

\section{Zdroje}

\begin{compactitem}
    \item \path{MSP_pred_01_Opakovani_Pravd-NP-NV.pdf}
    \item \path{MSP_pred_02_Opakovani_Statistika_Regrese.pdf}
\end{compactitem}

%%%%%%%%%%%%%%%%%%%%%%%%%%%%%%%%%%%%%%%%%%%%%%%%%%%%%%%%%%%%%%%%%%%%%%%%%%%%%%%%

\section{Náhodná proměnná}

\begin{compactitem}
    \item \textbf{Neformálně}; Náhodná proměnná je funkce, která přiřazuje každému elementárnímu náhodnému jevu nějakou (zpravidla číselnou) hodnotu (například při hodu mincí \uv{hlavě} nulu a \uv{orlu} jedničku). \begin{compactitem}
        \item Náhodnou veličinu lze jednoduše charakterizovat jako veličinu, jejíž hodnoty nelze před provedením pozorování jednoznačně určit, ale závisí na náhodě.
    \end{compactitem}

    \item \textbf{Formálně}; Nechť $\Omega$ je základní prostor a $(\Sigma, \Omega)$ je jevové pole. Pak zobrazení $X : \Omega \rightarrow \mathbb{R}$ se nazývá náhodná proměnná, pokud je měřitelné, tj.
    $$\forall x \in \mathbb{R} : \{ \omega \in \Omega ~|~ X(\omega) < x \} \in \Sigma$$

    \item Realizaci náhodné veličiny, tj. $X(\omega),~ \omega \in \Omega$ označíme $x$, pak \begin{compactitem}
        \item množinu $\{ \omega \in \Omega ~|~ X(\omega) < x \}$ zapisujeme jako $\{ X < x \}$,

        \item množinu $\{ \omega \in \Omega ~|~ X(\omega) = x \}$ zapisujeme jako $\{ X = x \}$.
    \end{compactitem}

    \item Obor hodnot náhodné proměnné $X$ se nazývá základní soubor a značí se $Z$.
    $$ Z = \{ x \in \mathbb{R} ~|~ x = X(\omega) \land \omega \in \Omega \} $$ \begin{compactitem}

        \item Náhodná proměnná se nazývá \textbf{diskrétní}, pokud $Z$ je nejvýše spočetná množina.

        \item Náhodná proměnná se nazývá \textbf{spojitá}, pokud $Z$ je nespočetná množina.
    \end{compactitem}
\end{compactitem}

\subsection{Distribuční funkce}

\begin{compactitem}
    \item Hodnota $P(\{ \omega \in \Omega ~|~ X(\omega) < x \})$ se nazývá distribuční funkce náhodné veličiny $X$ a značí se $F(x)$.

    \item Zkráceně zapisujeme jako $F(x) = P(X < x)$.

    \item Distribuční funkce udává, že hodnota náhodné proměnné je menší než zadaná hodnota.

    \item Platí: \begin{compactitem}

        \item $F(X)$ je neklesající,
        \item $F(X)$ je zleva spojitá,
        \item limity:
        $$ \lim_{x \rightarrow - \infty} F(x) = 0 $$
        $$ \lim_{x \rightarrow + \infty} F(x) = 1 $$
    \end{compactitem}

\end{compactitem}

Hustota pravděpodobnosti

Funkční charakteristiky

Číselné charakteristiky

Diskrétní náhodná proměnná

Spojitá náhodná proměnná

Rozdělení pravděpodobnosti
