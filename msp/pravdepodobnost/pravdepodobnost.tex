% VUT FIT MITAI
% MSZ 2021/2022
% Author: Vladimir Dusek
% Login: xdusek27

%%%%%%%%%%%%%%%%%%%%%%%%%%%%%%%%%%%%%%%%%%%%%%%%%%%%%%%%%%%%%%%%%%%%%%%%%%%%%%%%

% Path to figures
\graphicspath{{msp/pravdepodobnost/figures}}

%%%%%%%%%%%%%%%%%%%%%%%%%%%%%%%%%%%%%%%%%%%%%%%%%%%%%%%%%%%%%%%%%%%%%%%%%%%%%%%%

\chapter{MSP~--~Pravděpodobnost, podmíněná pravděpodobnost, nezávislost.}

%%%%%%%%%%%%%%%%%%%%%%%%%%%%%%%%%%%%%%%%%%%%%%%%%%%%%%%%%%%%%%%%%%%%%%%%%%%%%%%%

\section{Zdroje}

\begin{compactitem}
    \item \path{MSP_pred_01_Opakovani_Pravd-NP-NV.pdf}
    \item \path{SUR_2020-02-11.mp4}
    \item Wikipedia
\end{compactitem}

%%%%%%%%%%%%%%%%%%%%%%%%%%%%%%%%%%%%%%%%%%%%%%%%%%%%%%%%%%%%%%%%%%%%%%%%%%%%%%%%

\section{Popisná statistika}

\todo{todo}

%%%%%%%%%%%%%%%%%%%%%%%%%%%%%%%%%%%%%%%%%%%%%%%%%%%%%%%%%%%%%%%%%%%%%%%%%%%%%%%%

\section{Základy kombinatoriky}

\subsection{Variace}

\begin{compactitem}
    \item Variace $k$-té třídy z $n$ prvků je každá uspořádaná $k$-tice vytvořená z celkového počtu $n$ prvků, přičemž při výběru záleží na pořadí jednotlivých prvků.

    \item \textbf{Variace bez opakování} je $k$-členná skupina utvořená z daných $n$ prvků tak, že v nich záleží na pořadí a žádný z daných prvků se v ní neopakuje.
    $$ V(k, n) = \frac{n!}{(n-k)!} ~,~ k \leq n $$

    \item \textbf{Variace s opakováním} je uspořádaná $k$-tice z $n$ prvků sestavená tak, že každý se v ní vyskytuje nejvýše $k$-krát.
    $$ V'(k, n) = n^k ~,~ k < n $$
\end{compactitem}

\subsection{Permutace}

\begin{compactitem}
    \item Permutace $n$-prvkové množiny je uspořádaná $n$-tice obsahující každý prvek právě jednou, takže jednoznačně určuje jedno z možných uspořádání těchto prvků. Odtud (řídce užívané) české synonymum pro permutaci pořadí. Ekvivalentní definice je, že se jedná o $n$-prvkovou variaci z $n$ prvků.

    \item \textbf{Permutace bez opakování} -- Pokud se prvky ve výběru nemohou opakovat, pak počet všech možných pořadí je určen vztahem.
    $$ P(n) = n! $$

    \item \textbf{Permutace s opakováním} -- Pokud se prvky ve výběru mohou opakovat, pak počet permutací s opakováním z $n$ prvků je určen jako.
    $$ P'{(k_1,k_2,\ldots,k_n)} = \frac {(k_1+k_2+\ldots+k_n)!}{{k_1!}\cdot{k_2!}\cdot\ldots\cdot{k_n!}} $$
\end{compactitem}

\subsection{Kombinace}

\begin{compactitem}
    \item Kombinace je základní pojem z kombinatoriky, $k$-členná kombinace z $n$ prvků je skupina $k$ prvků, vybraná z $n$ různých prvků, u níž nezáleží na jejich pořadí. Od variace se liší tím, že je neuspořádaná.

    \item \textbf{Kombinace bez opakování} -- Počet kombinací $k$-té třídy z $n$-prvků bez opakování, neuspořádaných $k$-tic vybraných z těchto prvků tak, že se v ní každý vyskytuje nejvýše jednou, je
    $$ {\displaystyle C_{k}(n)={n \choose k}={n! \over k!(n-k)!}} $$

    \item \textbf{Kombinace s opakováním} -- Počet kombinací $k$-té třídy z $n$ prvků s opakováním, tzn. každý prvek se ve výběru může objevit vícekrát, je určen vztahem.
    $$ {\displaystyle C_{k}^{\prime }(n)={{(n+k-1)} \choose n-k}={{(n+k-1)} \choose k}={(n+k-1)! \over k!(n-1)!}}$$

    \item Kombinační čísla \begin{compactitem}
        \item Platí:
        $$ {n \choose k} = \frac{n!}{k! \cdot (n-k)!} $$
        $$ 1 = {0 \choose 0} = {n \choose 0} = {n \choose n} $$
        $$ {n \choose 1} = n$$
        $$ 0 \leq k \leq n$$
    \end{compactitem}
\end{compactitem}

%%%%%%%%%%%%%%%%%%%%%%%%%%%%%%%%%%%%%%%%%%%%%%%%%%%%%%%%%%%%%%%%%%%%%%%%%%%%%%%%

\section{Základy pravděpodobnosti}

\begin{compactitem}
    \item Experiment (také vědecký pokus) je soubor jednání a pozorování, jehož účelem je ověřit (verifikovat) nebo vyvrátit (falzifikovat) hypotézu nebo poznatek, které něco tvrdí o příčinných vztazích určitých fenoménů. \begin{compactitem}

        \item Deterministický pokus -- Vede vždy k jedinému možnému výsledku.

        \item Náhodný (stochastický) pokus -- Vede k jednomu z více možných výsledků.
    \end{compactitem}

    \item \textbf{Základní prostor} -- Množina všech možných výsledků pokusu, značíme $\Omega$.

    \item \textbf{Náhodný jev} -- Tvrzení o výsledku náhodného pokusu, o kterém lze po uskutečnění pokusu jednoznačně rozhodnout, zda je čí není pravdivé. Formálněji; nechť $A$ je náhodný jev, platí $A \subseteq \Omega$, tj. náhodným jevem rozumíme libovolnou podmnožinu základního prostoru.

    \item \textbf{Elementární náhodný jev} -- Náhodný jev $A$ se nazývá elementární náhodný jev, pokud neexistují různé náhodné jevy $B$, $C$, takové, že $A = B \cup C$.
\end{compactitem}

\subsection{Jevové pole}

\begin{compactitem}
    \item Jevové pole je dvojice $(\Sigma, \Omega)$, kde $\Sigma$ je množina náhodných jevů, resp. systém podmnožin základního prostoru $\Sigma \subseteq 2^{\Omega}$.

    \item Platí: \begin{compactenum}
        \item $ A \in \Sigma \Rightarrow \bar{A} \in \Sigma $
        \item $ A, B \in \Sigma \Rightarrow A \cup B, A \cap B, A \setminus B \in \Sigma $
        \item $ \emptyset \in \Sigma,~ \Omega \in \Sigma $
    \end{compactenum}

    \item Dále: \begin{compactitem}
        \item $\omega \in \Omega$ je elementární náhodný jev,
        \item $\Omega$ je jistý jev,
        \item $\emptyset$ je nemožný jev.
    \end{compactitem}

\end{compactitem}

\subsection{Kolmogorova axiomatická definice pravděpodobnosti}

\begin{compactitem}
    \item Nechť $(\Sigma, \Omega)$ je jevové pole, potom zobrazení $P : \Sigma \rightarrow \langle 0, 1 \rangle$ nazveme pravděpodobností na jevovém poli $(\Sigma, \Omega)$, pokud splňuje: \begin{compactitem}
        \item nezápornost,
        $$\forall A \in \Sigma ~:~ P(A) \geq 0$$
        \item normovanost,
        $$ P(\Omega) = 1 $$
        \item pravděpodobnost sjednocení disjunktních náhodných jevů je stejná jako součet jejich pravděpodobností.
        $$ \forall A_1, \ldots, A_n \in \Sigma : A_i \cap A_j = \emptyset \Rightarrow P \left( \bigcup_{i=1} A_i \right) = \sum_{i=1} P(A_i) $$
    \end{compactitem}

    \item Pro $A \in \Sigma$ nazýváme hodnotu $P(A)$ pravděpodobností jevu $A$.

    \item Trojici $ (\Omega, \Sigma, P) $ nazýváme pravděpodobnostní prostor.
\end{compactitem}

\subsection{Klasická (Laplaceova) pravděpodobnost}

\begin{compactitem}
    \item Předpokládejme: \begin{compactitem}
        \item Základní prostor $\Omega$ je konečná množina.
        \item Pro elementární jevy platí:
        $$\forall \omega \in \Omega : P(\omega) = \frac{1}{|\Omega|}$$
    \end{compactitem}

    \item Pak pravděpodobnost definujeme jako poměr počtu příznivých jevů ku počtu všech jevů.
    $$ \forall A \in 2^{\Omega} : P(A) = \frac{|A|}{|\Omega|} $$

    \item Konvence pro zároveň:
    $$ P(A, B) = P(A) \cdot P(B) $$
\end{compactitem}

\paragraph*{Příklad: házení šestistrannou kostkou}

\begin{compactitem}
    \item Základní prostor: $\Omega = \{ 1, 2, 3, 4, 5, 6 \}$
    \item Jevové pole: $(2^{\Omega}, \Omega)$.
    \item Elementární náhodné jevy: $ \{1\}, \{2\}, \{3\}, \{4\}, \{5\}, \{6\} $
    \item Náhodný jev \uv{sudé číslo}: $A = \{ 2, 4, 6 \}$
    $$ P(A) = \frac{|A|}{|\Omega|} = \frac{3}{6} = 0,5 $$
    \item Náhodný jev \uv{číslo větší než 4}: $B = \{ 5, 6 \}$
    $$ P(B) = \frac{|B|}{|\Omega|} = \frac{2}{6} = 0,\bar{3} $$
\end{compactitem}

\subsection{Geometrická pravděpodobnost}

\begin{compactitem}
    \item Rozšíření klasické pravděpodobnosti pro nekonečný základní prostor $\Omega$.

    \item Zde je definice pravděpodobnosti založena na porovnání objemů, ploch či délek geometrických útvarů.

    \item Uvažujme dvojrozměrný prostor, výpočet pravděpodobnosti pak vypadá následovně:
    $$ P(A) = \frac{A_S}{\Omega_S}$$
    Kde $A_S$ je obsah plochy reprezentující jev $A$ a $\Omega_S$ je obsah plochy reprezentující všechny možné výsledky.
\end{compactitem}

%%%%%%%%%%%%%%%%%%%%%%%%%%%%%%%%%%%%%%%%%%%%%%%%%%%%%%%%%%%%%%%%%%%%%%%%%%%%%%%%

\section{Podmíněná pravděpodobnost}

\subsection{Podmíněná pravděpodobnost}

\begin{compactitem}
    \item Nechť $(\Sigma, \Omega)$ je jevové pole, $A, B \in \Sigma$ jsou náhodné jevy a platí $P(B) > 0$.

    \item Podmíněná pravděpodobnost jevu $A$ vzhledem k jevu $B$ pak je:
    $$ P(A ~|~ B) = \frac{P(A \cap B)}{P(B)} $$

    \item Podmíněná pravděpodobnost udává s jakou pravděpodobností nastane jev $A$, když už nastal jev $B$.
\end{compactitem}

\subsection{Věta o úplné pravděpodobnosti}

\begin{compactitem}
    \item Nechť $B_i \in 2^{\Omega} ~,~ i \in \{ 1, 2, \ldots, n \}$ je rozklad základního prostoru $\Omega$, platí
    $$ \bigcup_{i=1}^n B_i = \Omega ~\land~ i, j \in \{ 1, 2, \ldots, n \} : B_i \not= B_j \Rightarrow B_i \cap B_j = \emptyset $$

    \item Nechť $A \in 2^{\Omega}$ je náhodný jev, platí:
    $$ P(A) = \sum_{i=1}^n P(B_i) \cdot P(A ~|~ B_i) $$
\end{compactitem}

\subsection{Bayesova věta}

\begin{compactitem}
    \item Nechť $B_i \in 2^{\Omega} ~,~ i \in \{ 1, 2, \ldots, n \}$ je rozklad základního prostoru $\Omega$, platí
    $$ \bigcup_{i=1}^n B_i = \Omega ~\land~ i, j \in \{ 1, 2, \ldots, n \} : B_i \not= B_j \Rightarrow B_i \cap B_j = \emptyset $$

    \item Nechť $A \in 2^{\Omega}$ je náhodný jev, o němž víme, že již nastal.
    $$ P(B_i ~|~ A) = \frac{P(B_i) \cdot P(A ~|~ B_i)}{P(A)}$$

    \item Pravděpodobnost $P(A)$ lze spočítat na základě věty o úplné pravděpodobnosti.
\end{compactitem}

\paragraph*{Příklad: V populaci užívá drogy $2\,\%$ lidí. Test na drogy má $3\,\%$ šanci na falešnou pozitivitu $4\,\%$ na falešnou negativitu. Pokud vyjde člověku test pozitivní, jaká je pravděpodobnost, že skutečně bere drogy?}

\begin{compactitem}
    \item Nechť $A$ jsou užívalé drog a $B$ jsou neuživatelé drog, pak platí:
    $$ P(A) = \frac{2}{100} = 0,02 ~~~~~,~~~~~ P(B) = \frac{98}{100} = 0,98$$

    \item Pozitivní výsledek testu na drogy označme jako $P$ a negativní jako $N$, pak platí:
    $$ P(P ~|~ A) = \frac{97}{100} = 0,97 ~~~~~,~~~~~ P(P ~|~ B) = \frac{3}{100} = 0,03 $$
    $$ P(N ~|~ A) = \frac{4}{100} = 0,04 ~~~~~,~~~~~ P(N ~|~ B) = \frac{96}{100} = 0,96 $$

    \item Na základě věty o úplně pravděpodobnosti:
    $$ P(P) = P(A) \cdot P(P~|~A) + P(B) \cdot P(P~|~B) = 0,02 \cdot 0,97 + 0,98 \cdot 0,03 = 0,0488$$

    \item Na základě bayesovy věty:
    \begin{equation}
        \begin{split}
            P(A ~|~ P) &= \frac{ P(A) \cdot P(P ~|~ A) }{P(P)} = \frac{ 0,02 \cdot 0,97 }{0,0488} \approx 39,75\,\%
        \end{split}
    \end{equation}
\end{compactitem}

\paragraph*{Příklad: Jak odhadnu pravděpodobnost, že dostanu infarkt, když budu tlustý, pokud mám statistiky pořízené na velkém vzorku populace, kde je u každého člověka záznam o tom, zda byl tlustý a zda prodělal infarkt.}

\begin{compactitem}
    \item Na základě věty o úplně pravděpodobnosti:
    $$ P(tlusty) = P(infarkt) \cdot P(tlusty ~|~ infarkt) + P(neInfarkt) \cdot P(tlusty ~|~ neInfarkt)$$

    \item Na základě bayesovy věty:
    $$ P(infarkt ~|~ tlusty) = \frac{P(infarkt) \cdot P(tlusty ~|~ infarkt)}{P(tlusty)} $$
\end{compactitem}

\subsection{Nezávislost}

\begin{compactitem}
    \item Řekneme, že jevy $A$ a $B$ jsou nezávislé, pokud pravděpodobnost jevu $A$ nezávisí na výskytu jevu $B$.

    \item Formálně: nechť $A, B \in 2^{\Omega}$ jsou náhodné jevy, tyto jevy nazveme nezávislé, pokud platí:
    $$ P(A ~|~ B) = P(A) ~\lor~ P(B ~|~ A) = P(B)$$

    \item Lze upravit:
    $$ P(A ~|~ B) = P(A) $$
    $$ \frac{P(A \cap B)}{P(B)} = P(A) $$
    $$ P(A \cap B) = P(A) \cdot P(B) $$

\end{compactitem}

\paragraph*{Příklad: Házení kostkou a (ne)závislé jevy.}

\begin{compactitem}
    \item $\Omega = \{ 1,2,3,4,5,6 \}$
    \item Příklad 1: \begin{compactitem}
        \item $A = \{ 2,4,6 \}$
        \item $B = \{ 5,6 \}$
        $$ P(A \cap B) = P(A) \cdot P(B) ~\Rightarrow~ \frac{1}{6} = \frac{1}{3} \cdot \frac{1}{2} $$
        \item Jevy $A$ a $B$ jsou nezávislé.
    \end{compactitem}

    \item Příklad 2: \begin{compactitem}
        \item $C = \{ 1,3,5 \}$
        \item $D = \{ 3,5 \}$
        $$ P(C \cap D) = P(C) \cdot P(D) ~\Rightarrow~ \frac{1}{3} \not= \frac{1}{2} \cdot \frac{1}{3} $$
        \item Jevy $A$ a $B$ jsou závislé.
    \end{compactitem}
\end{compactitem}

\subsection{Sum Rule}

\begin{compactitem}
    \item Nechť $\Omega$ je základní prostor a $A \in 2^{\Omega}$ je náhodný jev, pak platí tzv. \textit{sum rule}:
    $$ P(A) = \sum_{B \in 2^{\Omega}} P(A) \cdot P(B)$$
\end{compactitem}

\subsection{Product rule}

\begin{compactitem}
    \item Nechť $\Omega$ je základní prostor a $A, B \in 2^{\Omega}$ jsou náhodné jevy, pak platí tzv. \textit{product rule}:
    $$ P(A, B) = P(A ~|~ B) \cdot P(B) = P(B ~|~ A) \cdot P(A) $$
\end{compactitem}
