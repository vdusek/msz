% VUT FIT MITAI
% MSZ 2021/2022
% Author: Vladimir Dusek
% Login: xdusek27

%%%%%%%%%%%%%%%%%%%%%%%%%%%%%%%%%%%%%%%%%%%%%%%%%%%%%%%%%%%%%%%%%%%%%%%%%%%%%%%%

% Path to figures
\graphicspath{{msp/pravdepodobnost/figures}}

%%%%%%%%%%%%%%%%%%%%%%%%%%%%%%%%%%%%%%%%%%%%%%%%%%%%%%%%%%%%%%%%%%%%%%%%%%%%%%%%

\chapter{MSP~--~Pravděpodobnost, podmíněná pravděpodobnost, nezávislost.}

%%%%%%%%%%%%%%%%%%%%%%%%%%%%%%%%%%%%%%%%%%%%%%%%%%%%%%%%%%%%%%%%%%%%%%%%%%%%%%%%

\section{Zdroje}

\begin{compactitem}
    \item \path{MSP_pred_01_Opakovani_Pravd-NP-NV.pdf}
\end{compactitem}

%%%%%%%%%%%%%%%%%%%%%%%%%%%%%%%%%%%%%%%%%%%%%%%%%%%%%%%%%%%%%%%%%%%%%%%%%%%%%%%%

\section{Úvod a kontext}

\begin{compactitem}
    \item Experiment (česky též vědecký pokus) je soubor jednání a pozorování, jehož účelem je ověřit (verifikovat) nebo vyvrátit (falzifikovat) hypotézu nebo poznatek, které něco tvrdí o příčinných vztazích určitých fenoménů.

    \item Deterministický pokus -- vede vždy k jedinému možnému výsledku.

    \item Náhodný (stochastický) pokus -- vede k jednomu z více možných výsledků.

    \item Náhodný jev -- je tvrzení o výsledku náhodného pokusu, o kterém lze po uskutečnění pokusu jednoznačně rozhodnout, zda je čí není pravdivé.

    \item Základní prostor $\Omega$ je množina všech možných výsledků pokusu.

    \item Náhodným jevem $A$ pak rozumíme libovolnou podmnožinu základního prostoru $A \subseteq \Omega$.

    \item Náhodný jev $A$ se nazývá elementární náhodný jev, pokud neexistují různé náhodné jevy $B$, $C$, takové, že $A = B \cup C$.

\end{compactitem}

%%%%%%%%%%%%%%%%%%%%%%%%%%%%%%%%%%%%%%%%%%%%%%%%%%%%%%%%%%%%%%%%%%%%%%%%%%%%%%%%

\section{Základy pravděpodobnosti}

\subsection{Jevové pole}

\begin{compactitem}
    \item Jevové pole je dvojice $(\Sigma, \Omega)$, kde $\Sigma$ je množina náhodných jevů, resp. systém podmnožin základního prostoru $\Sigma \subseteq 2^{\Omega}$.

    \item Platí: \begin{compactenum}
        \item $ A \in \Sigma \Rightarrow \bar{A} \in \Sigma $
        \item $ A, B \in \Sigma \Rightarrow A \cup B, A \cap B, A \setminus B \in \Sigma $
        \item $ \emptyset \in \Sigma,~ \Omega \in \Sigma $
    \end{compactenum}

    \item Dále \begin{compactitem}
        \item $\omega \in \Omega$ je elementární náhodný jev,
        \item $\Omega$ je jistý jev,
        \item $\emptyset$ je nemožný jev.
    \end{compactitem}

\end{compactitem}

\subsection{Kolmogorova axiomatická definice pravděpodobnosti}

\begin{compactitem}
    \item Nechť $(\Sigma, \Omega)$ je jevové pole, potom zobrazení $P : \Sigma \rightarrow \langle 0, 1 \rangle$ nazveme pravděpodobností na jevovém poli $(\Sigma, \Omega)$, pokud splňuje: \begin{compactitem}
        \item nezápornost,
        $$\forall A \in \Sigma ~:~ P(A) \geq 0$$
        \item normovanost,
        $$ P(\Omega) = 1 $$
        \item pravděpodobnost sjednocení disjunktních náhodných jevů je stejná jako součet jejich pravděpodobností.
        $$ \forall A_1, \ldots, A_n \in \Sigma : A_i \cap A_j = \emptyset \Rightarrow P \left( \bigcup_{i=1} A_i \right) = \sum_{i=1} P(A_i) $$
    \end{compactitem}

    \item Pro $A \in \Sigma$ nazýváme hodnotu $P(A)$ pravděpodobností jevu $A$.

    \item Trojici $ (\Omega, \Sigma, P) $ nazýváme pravděpodobnostní prostor.
\end{compactitem}

\subsection{Klasická pravděpodobnost}

\begin{compactitem}
    \item Předpokládejme: \begin{compactitem}
        \item Základní prostor $\Omega$ je konečná množina.
        \item Pro elementární jevy platí:
        $$\forall \omega \in \Omega : P(\omega) = \frac{1}{|\Omega|}$$
    \end{compactitem}

    \item Pak pravděpodobnost definujeme jako poměr počtu příznivých jevů ku počtu všech jevů.
    $$ \forall A \in 2^{\Omega} : P(A) = \frac{|A|}{|\Omega|} $$

    \item Konvence pro zároveň:
    $$ P(A, B) = P(A) \cdot P(B) $$
\end{compactitem}

\paragraph*{Příklad: házení šestistrannou kostkou}

\begin{compactitem}
    \item Základní prostor: $\Omega = \{ 1, 2, 3, 4, 5, 6 \}$
    \item Jevové pole: $(2^{\Omega}, \Omega)$.
    \item Elementární náhodné jevy: $ \{1\}, \{2\}, \{3\}, \{4\}, \{5\}, \{6\} $
    \item Náhodný jev \uv{sudé číslo}: $A = \{ 2, 4, 6 \}$
    $$ P(A) = \frac{|A|}{|\Omega|} = \frac{3}{6} = 0,5 $$
    \item Náhodný jev \uv{číslo větší než 4}: $B = \{ 5, 6 \}$
    $$ P(B) = \frac{|B|}{|\Omega|} = \frac{2}{6} = 0,\bar{3} $$
\end{compactitem}

\subsection{Geometrická pravděpodobnost}

\begin{compactitem}
    \item Rozšíření klasické pravděpodobnosti pro nekonečný základní prostor $\Omega$.

    \item Zde je definice pravděpodobnosti založena na porovnání objemů, ploch či délek geometrických útvarů.

    \item Uvažujme dvojrozměrný prostor, výpočet pravděpodobnosti pak vypadá následovně:
    $$ P(A) = \frac{A_S}{\Omega_S}$$
    Kde $A_S$ je obsah plochy reprezentující jev $A$ a $\Omega_S$ je obsah plochy reprezentující všechny možné výsledky.
\end{compactitem}

\subsection{Sum Rule}

\begin{compactitem}
    \item Nechť $\Omega$ je základní prostor a $A \in 2^{\Omega}$ je náhodný jev, pak platí tzv. \textit{sum rule}:
    $$ P(A) = \sum_{B \in 2^{\Omega}} P(A) \cdot P(B)$$
\end{compactitem}

%%%%%%%%%%%%%%%%%%%%%%%%%%%%%%%%%%%%%%%%%%%%%%%%%%%%%%%%%%%%%%%%%%%%%%%%%%%%%%%%

\section{Podmíněná pravděpodobnost}

\subsection{Podmíněná pravděpodobnost}

\begin{compactitem}
    \item Nechť $(\Sigma, \Omega)$ je jevové pole, $A, B \in \Sigma$ jsou náhodné jevy a platí $P(B) > 0$.

    \item Podmíněná pravděpodobnost jevu $A$ vzhledem k jevu $B$ pak je:
    $$ P(A ~|~ B) = \frac{P(A \cap B)}{P(B)} $$

    \item Podmíněná pravděpodobnost udává s jakou pravděpodobností nastane jev $A$, když už nastal jev $B$.
\end{compactitem}

\subsection{Věta o úplné pravděpodobnosti}

\begin{compactitem}
    \item Nechť $B_i \in 2^{\Omega} ~,~ i \in \{ 1, 2, \ldots, n \}$ je rozklad základního prostoru $\Omega$, platí
    $$ \bigcup_{i=1}^n B_i = \Omega ~\land~ i, j \in \{ 1, 2, \ldots, n \} : B_i \not= B_j \Rightarrow B_i \cap B_j = \emptyset $$

    \item Nechť $A \in 2^{\Omega}$ je náhodný jev, platí:
    $$ P(A) = \sum_{i=1}^n P(B_i) \cdot P(A ~|~ B_i) $$
\end{compactitem}

\subsection{Product rule}

\begin{compactitem}
    \item Nechť $\Omega$ je základní prostor a $A, B \in 2^{\Omega}$ jsou náhodné jevy, pak platí tzv. \textit{product rule}:
    $$ P(A, B) = P(A ~|~ B) \cdot P(B) = P(B ~|~ A) \cdot P(A) $$
\end{compactitem}

\subsection{Bayesova věta}

\begin{compactitem}
    \item Nechť $B_i \in 2^{\Omega} ~,~ i \in \{ 1, 2, \ldots, n \}$ je rozklad základního prostoru $\Omega$, platí
    $$ \bigcup_{i=1}^n B_i = \Omega ~\land~ i, j \in \{ 1, 2, \ldots, n \} : B_i \not= B_j \Rightarrow B_i \cap B_j = \emptyset $$

    \item Nechť $A \in 2^{\Omega}$ je náhodný jev, o němž víme, že již nastal.
    $$ P(B_i ~|~ A) = \frac{P(B_i) \cdot P(A ~|~ B_i)}{P(A)} = \frac{P(B_i) \cdot P(A ~|~ B_i)}{\sum_{j=1}^n P(B_j) \cdot P(A ~|~ B_j)} $$
\end{compactitem}

\subsection{Nezávislost}

\begin{compactitem}
    \item Nechť $A, B \in 2^{\Omega}$ jsou náhodné jevy. Tyto jevy nazveme nezávislé, pokud platí:
    $$ P(A ~|~ B) = P(A) ~\lor~ P(B ~|~ A) = P(B)$$

    \item Alternativně lze zapsat jako:
    $$ P(A \cap B) = P(A) \cdot P(B) $$
\end{compactitem}

\paragraph*{Příklad:} \todo{Todo: priklad na nezavislost}

\paragraph*{Příklad: V populaci užívá drogy $2\,\%$ lidí. Test na drogy má $3\,\%$ šanci na falešnou pozitivitu $4\,\%$ na falešnou negativitu. Pokud vyjde člověku test pozitivní, jaká je pravděpodobnost, že skutečně bere drogy?}

\begin{compactitem}
    \item Nechť $A$ jsou užívalé drog a $B$ jsou neuživatelé drog, pak platí:
    $$ P(A) = \frac{2}{100} = 0,02 ~~~~~,~~~~~ P(B) = \frac{98}{100} = 0,98$$

    \item Pozitivní výsledek testu na drogy označme jako $P$ a negativní jako $N$, pak platí:
    $$ P(P ~|~ A) = \frac{97}{100} = 0,97 ~~~~~,~~~~~ P(P ~|~ B) = \frac{3}{100} = 0,03 $$
    $$ P(N ~|~ A) = \frac{4}{100} = 0,04 ~~~~~,~~~~~ P(N ~|~ B) = \frac{96}{100} = 0,96 $$

    \item Otázka zní:
    \begin{equation}
        \begin{split}
            P(A ~|~ P) &= \frac{ P(A) \cdot P(P ~|~ A) }{P(P)} = \\
            \\
            &= \frac{ P(A) \cdot P(P ~|~ A) }{P(A) \cdot P(P~|~A) + P(B) \cdot P(P~|~B)} = \\
            \\
            &= \frac{ 0,02 \cdot 0,97 }{0,02 \cdot 0,97 + 0,98 \cdot 0,03} \approx \\
            \\
            &\approx 39,75\,\%
        \end{split}
    \end{equation}
\end{compactitem}

\paragraph*{Příklad: Jak odhadnu pravděpodobnost, že dostanu infarkt, když budu tlustý, pokud mám statistiky pořízené na velkém vzorku populace, kde je u každého člověka záznam o tom, zda byl tlustý a zda prodělal infarkt.}

\begin{equation}
    \begin{split}
        &P(infarkt ~|~ tlusty) = \frac{P(infarkt) \cdot P(tlusty ~|~ infarkt)}{P(tlusty)} = \\
        \\
        &= \frac{P(infarkt) \cdot P(tlusty ~|~ infarkt)}{P(infarkt) \cdot P(tlusty ~|~ infarkt) + P(neInfarkt) \cdot P(tlusty ~|~ neInfarkt)}
    \end{split}
\end{equation}
