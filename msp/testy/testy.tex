% VUT FIT MITAI
% MSZ 2021/2022
% Author: Vladimir Dusek
% Login: xdusek27

%%%%%%%%%%%%%%%%%%%%%%%%%%%%%%%%%%%%%%%%%%%%%%%%%%%%%%%%%%%%%%%%%%%%%%%%%%%%%%%%

% Path to figures
\graphicspath{{msp/testy/figures}}

%%%%%%%%%%%%%%%%%%%%%%%%%%%%%%%%%%%%%%%%%%%%%%%%%%%%%%%%%%%%%%%%%%%%%%%%%%%%%%%%

\chapter{MSP~--~Vícevýběrové testy, testy o rozdělení, testy dobré shody.}

%%%%%%%%%%%%%%%%%%%%%%%%%%%%%%%%%%%%%%%%%%%%%%%%%%%%%%%%%%%%%%%%%%%%%%%%%%%%%%%%

\textit{Pozn. tato otázka je zpracovaná pouze stručně.}

\section{Zdroje}

\begin{compactitem}
    \item \path{MSP_pred_02_Opakovani_Statistika_Regrese.}pdf
    \item \path{MSP_pred_04_ANOVA.pdf}
    \item \path{MSP_pred_08_Testy_DS_Testy_rozdeleni.pdf}
    \item Wikipedia
\end{compactitem}

%%%%%%%%%%%%%%%%%%%%%%%%%%%%%%%%%%%%%%%%%%%%%%%%%%%%%%%%%%%%%%%%%%%%%%%%%%%%%%%%

\section{ANOVA}

Statistické metody, které umožňují provádět vícenásobné porovnávání středních hodnot, jsou soustředěny pod souhrnným názvem analýza rozptylu (ANOVA -- Analysis of Variance). Tato metoda je založena na hodnocení vztahů mezi rozptyly porovnávaných výběrových souborů (testování shody středních hodnot se převádí na testování shody dvou rozptylů).

\subsection{Jednofaktorová ANOVA}

\todo{todo}

\subsection{Dvoufaktorová ANOVA}

V praxi se často setkáváme s pokusy, kdy sledujeme více působících faktorů, např. vliv krmení a plemene, vliv léku v různých stádiích onemocnění, vliv živné půdy a způsobu kultivace na růst zárodků, vliv různých druhů antibiotik a jejich dávky apod. Pokud zkoumáme vliv dvou a více faktorů působících na závisle proměnnou, hovoříme o vícefaktorové analýze rozptylu.

%%%%%%%%%%%%%%%%%%%%%%%%%%%%%%%%%%%%%%%%%%%%%%%%%%%%%%%%%%%%%%%%%%%%%%%%%%%%%%%%

\section{Testy dobré shody}

Testy dobré shody vycházejí z porovnání teoretické pravděpodobnosti a odhadnuté pravděpodobnosti pomocí relativních četností u náhodné veličiny, která může nabývat konečného počtu možností.

\todo{todo}

%%%%%%%%%%%%%%%%%%%%%%%%%%%%%%%%%%%%%%%%%%%%%%%%%%%%%%%%%%%%%%%%%%%%%%%%%%%%%%%%

\section{Testy o rozdělení}

Lze provést pomocí testu dobré shody na tříděném statistickém souboru.

\todo{todo}
