% VUT FIT MITAI
% MSZ 2021/2022
% Author: Vladimir Dusek
% Login: xdusek27

%%%%%%%%%%%%%%%%%%%%%%%%%%%%%%%%%%%%%%%%%%%%%%%%%%%%%%%%%%%%%%%%%%%%%%%%%%%%%%%%

% Path to figures
\graphicspath{{tin/konecne_automaty/figures}}

%%%%%%%%%%%%%%%%%%%%%%%%%%%%%%%%%%%%%%%%%%%%%%%%%%%%%%%%%%%%%%%%%%%%%%%%%%%%%%%%

\chapter{TIN~--~Konečné automaty (jazyky přijímané KA, varianty KA, minimalizace KA, Mihill-Nerodova věta).}

%%%%%%%%%%%%%%%%%%%%%%%%%%%%%%%%%%%%%%%%%%%%%%%%%%%%%%%%%%%%%%%%%%%%%%%%%%%%%%%%

\section{Zdroje}

\begin{compactitem}
    \item \path{tin_2021_merged.pdf}
    \item \path{TIN_2020-09-22.mp4}
    \item \path{TIN_2020-09-29.mp4}
    \item \path{TIN_2020-10-06.mp4}
\end{compactitem}

%%%%%%%%%%%%%%%%%%%%%%%%%%%%%%%%%%%%%%%%%%%%%%%%%%%%%%%%%%%%%%%%%%%%%%%%%%%%%%%%

\section{Konečný automat}

Konečné automaty dokáží přijímat regulární jazyky.

\paragraph*{Konečný automat} Konečný automat (KA) je pětice $M = (Q, \Sigma, \delta, q_0, F)$, kde \begin{compactitem}
    \item $Q$ je konečná množina stavů;
    \item $\Sigma$ je vstupní abeceda;
    \item $\delta$ je přechodová funkce (parciální funkce), \begin{compactitem}
        \item $\delta : Q \times \Sigma \rightarrow 2^{Q}$;
    \end{compactitem}
    \item $q_0$ je výchozí stav, \begin{compactitem}
        \item $q_0 \in Q$;
    \end{compactitem}
    \item $F$ je množina koncových stavů, \begin{compactitem}
        \item $F \subseteq Q$;
    \end{compactitem}
\end{compactitem}

\paragraph*{Konfigurace KA} Konfigurace KA je dvojice $(q, w) \in Q \times \Sigma^*$, kde \begin{compactitem}
    \item $q$ je aktuální stav;
    \item $w$ je nezpracovaná část vstupního řetězce.
\end{compactitem}

\paragraph*{Počáteční konfigurace KA} Počáteční konfigurace je taková konfigurace $(q, w)$, kde $w \in \Sigma^*$ je vstupní řetězec a $q \in Q$ je výchozí stav.

\paragraph*{Finální konfigurace KA} Finální konfigurace je taková konfigurace $(q, w)$, kde $w = \epsilon$ je $q \in F$.

\paragraph*{Přechod KA} Přechod KA je binární relace (značíme $\vdash$) na množině konfigurací $\vdash ~ \subseteq (Q \times \Sigma^*)^2$, taková, že $$ \forall q, q' \in Q ~ \forall w, w' \in \Sigma^* : (q, w) \vdash (q', w') \Rightarrow  \exists a \in \Sigma \land aw = w' \land w' \in \delta(w, a)$$

\paragraph*{Jazyk přijímaný KA} Mějme KA $M = (Q, \Sigma, \delta, q_0, F)$ a jazyk $L(M)$, který je přijímaný KA $M$. $$ L(M) = \{ w ~|~ w \in \Sigma^* \land (q_0, w) \vdash^* (q_f, \epsilon) \land q_f \in F \}$$

\paragraph*{Dosažitelné a nedosažitelné stavy} Mějme KA $M = (Q, \Sigma, \delta, q_0, F)$. Stav $q \in Q$ je dosažitelný, pokud platí $\exists w \in \Sigma^* : (q_0, w) \vdash^* (q, \epsilon)$. Stav je nedosažitelný, pokud není dosažitelný.

\paragraph*{Relace nerozlišitelnosti} Relace nerozlišitelnosti je binární relace nad množinou stavů. Stavy $p, q \in Q$ jsou rozlišitelné, jestliže existuje řetězec $w \in \Sigma^*$, pro který KA ze stavu $q$ skončí v nějakém $q_f \in F$, ale ze stavu $p$ nikoliv. Dva stavy jsou nerozlišitelné, pokud nejsou rozlišitelné. Formálně: nechť $p, q \in Q$ jsou rozlišitelné, pak platí $$ \forall w \in \Sigma^* : (p, w) \vdash^* (p', \epsilon) \land (q, w) \vdash^* (q', \epsilon) \land ( (p' \in F \land q' \not\in F) \lor (p' \not\in F \land q' \in F) ) $$

\paragraph*{k-nerozlišitelnost} \todo{todo}

%%%%%%%%%%%%%%%%%%%%%%%%%%%%%%%%%%%%%%%%%%%%%%%%%%%%%%%%%%%%%%%%%%%%%%%%%%%%%%%%

\section{Varianty konečného automatu}

Všechny varianty konečného automatu mají stejnou vyjadřovací sílu (jsou mezi sebou převoditelné).

\paragraph*{Nedeterministický konečný automat} Nedeterministický konečný automat (NKA) je výchozí konečný automat (NKA = KA).

\paragraph*{Deterministický konečný automat} Deterministický konečný automat (DKA) se od NKA liší pouze tvarem přechodové funkce. Formálně: \begin{compactitem}
    \item $\delta$ je přechodová funkce (parciální funkce), \begin{compactitem}
        \item $\delta : Q \times \Sigma \rightarrow Q$
    \end{compactitem}
\end{compactitem}

\paragraph*{Rozšířený konečný automat} Rozšířený konečný automat (RKA) se od NKA liší pouze tvarem přechodové funkce. Rozšiřuje ji o tzv. epsilon přechody. Formálně: \begin{compactitem}
    \item $\delta$ je přechodová funkce (parciální funkce), \begin{compactitem}
        \item $\delta : Q \times (\Sigma \cup \{ \epsilon \}) \rightarrow 2^Q$
    \end{compactitem}
\end{compactitem}

\paragraph*{Úplně definovaný konečný automat} Úplně definovaný konečný automat se od NKA liší pouze tvarem přechodové funkce. Formálně: \begin{compactitem}
    \item $\delta$ je přechodová funkce (totální funkce), \begin{compactitem}
        \item $\delta : Q \times \Sigma \rightarrow 2^Q$
        \item $\forall q \in Q ~ \forall a \in \Sigma : \delta(q, a) \in Q$
    \end{compactitem}
\end{compactitem}

\paragraph*{Redukovaný úplně definovaný deterministický konečný automat} Úplně definovaný DKA nazýváme redukovaný, jestliže žádný $q \in Q$ není nedosažitelný a žádná dvojice $p, q \in Q$ není nerozlišitelná.

\subsection{Převod NKA na DKA}

\begin{figure}[H]
    \centering
    \includegraphics[width=0.75\linewidth]{nka.pdf}
    \caption{Nedeterministický konečný automat.}
\end{figure}

\begin{figure}[H]
    \centering
    \includegraphics[width=0.75\linewidth]{nka_to_dka.pdf}
    \caption{Algoritmus převodu NKA na DKA.}
\end{figure}

\begin{figure}[H]
    \centering
    \includegraphics[width=0.75\linewidth]{dka_uplny.pdf}
    \caption{Deterministický konečný automat.}
\end{figure}

\subsection{Převod RKA na DKA}

\todo{todo}

%%%%%%%%%%%%%%%%%%%%%%%%%%%%%%%%%%%%%%%%%%%%%%%%%%%%%%%%%%%%%%%%%%%%%%%%%%%%%%%%

\section{Minimalizace konečného automatu}

\begin{compactitem}
    \item Proč? Zrychlení vykonávání konečného automatu.
    \item Jak? Odstraněním nepotřebných přechodů a zmenšením počtu stavů.
\end{compactitem}

\begin{figure}[H]
    \centering
    \includegraphics[width=0.75\linewidth]{dka_uplny.pdf}
    \caption{Deterministický konečný automat.}
\end{figure}

\subsection{Eliminace nedosažitelných stavů}

\begin{figure}[H]
    \centering
    \includegraphics[width=0.9\linewidth]{eliminace_nedosazitelnych_stavu.pdf}
    \caption{Algoritmus eliminace nedosažitelných stavů.}
\end{figure}

\begin{figure}[H]
    \centering
    \includegraphics[width=0.75\linewidth]{dka.pdf}
    \caption{Deterministický konečný automat bez nedosažitelných stavů.}
\end{figure}

\subsection{Převod DKA na úplně definovaný}

\begin{compactitem}
    \item Vstup: DKA $M = (Q, \Sigma, \delta, q_0, F)$
    \item Výstup: úplně definovaný DKA $M'$
    \item Metoda: \begin{compactenum}
        \item $\forall q \in Q ~ \forall a \in \Sigma : \delta(q, a) = \emptyset \Rightarrow \delta'(q, a) = q_s$
        \item $\forall a \in \Sigma : \delta'(q_s, a) = q_s$
        \item $M' = (Q \cup \{ q_s \}, \Sigma, \delta \cup \delta', q_o, F)$
    \end{compactenum}
    \item $q_s$ je tzv. \textit{sink} stav.
\end{compactitem}

\begin{figure}[H]
    \centering
    \includegraphics[width=0.75\linewidth]{dka_uplny_sink.pdf}
    \caption{Úplně definovaný deterministický konečný automat.}
\end{figure}

\subsection{Převod úplně definovaného DKA na redukovaný}

\begin{figure}[H]
    \centering
    \includegraphics[width=0.9\linewidth]{prevod_na_redukovany_dka.pdf}
    \caption{Algoritmus převodu úplně definovaného DKA na redukovaný.}
\end{figure}

\subsection{Odstranění \textit{sink} stavu}

\begin{compactitem}
    \item Vstup: DKA $M = (Q, \Sigma, \delta, q_0, F)$
    \item Výstup: DKA $M'$ bez \textit{sink} stavu
    \item Metoda: \begin{compactenum}
        \item $Q' = \{ q ~|~ q \in Q \land w \in \Sigma^* \land q_f \in F \land (q, w) \vdash^* (q_f, \epsilon) \}$
        \item $\delta' = \{ (p, a, q) ~|~ \delta(p, a) = q \land q \not\in Q - Q' \}$
        \item $M' = \{ Q', \Sigma, \delta', q_0, F \}$
    \end{compactenum}
\end{compactitem}

\begin{figure}[H]
    \centering
    \includegraphics[width=0.75\linewidth]{dka.pdf}
    \caption{Úplně definovaný deterministický konečný automat bez \textit{sink} stavu.}
\end{figure}

%%%%%%%%%%%%%%%%%%%%%%%%%%%%%%%%%%%%%%%%%%%%%%%%%%%%%%%%%%%%%%%%%%%%%%%%%%%%%%%%

\section{Mihill-Nerodova věta}

\todo{todo}
