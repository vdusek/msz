% VUT FIT MITAI
% MSZ 2021/2022
% Author: Vladimir Dusek
% Login: xdusek27

%%%%%%%%%%%%%%%%%%%%%%%%%%%%%%%%%%%%%%%%%%%%%%%%%%%%%%%%%%%%%%%%%%%%%%%%%%%%%%%%

% Path to figures
\graphicspath{{tin/chomskeho_hierarchie/figures}}

%%%%%%%%%%%%%%%%%%%%%%%%%%%%%%%%%%%%%%%%%%%%%%%%%%%%%%%%%%%%%%%%%%%%%%%%%%%%%%%%

\chapter{TIN~--~Konečné automaty (jazyky přijímané KA, varianty KA, minimalizace KA, Mihill-Nerodova věta).}

%%%%%%%%%%%%%%%%%%%%%%%%%%%%%%%%%%%%%%%%%%%%%%%%%%%%%%%%%%%%%%%%%%%%%%%%%%%%%%%%

\section{Zdroje}

\begin{compactitem}
    \item \path{tin_2021_merged.pdf}
    \item \path{TIN_2020-09-22.mp4}
    \item \path{TIN_2020-09-29.mp4}
\end{compactitem}

%%%%%%%%%%%%%%%%%%%%%%%%%%%%%%%%%%%%%%%%%%%%%%%%%%%%%%%%%%%%%%%%%%%%%%%%%%%%%%%%

\section{Konečný automat}

\paragraph*{Konečný automat} Konečný automat (KA) je pětice $M = (Q, \Sigma, \delta, s, F)$, kde \begin{compactitem}
    \item $Q$ je konečná množina stavů;
    \item $\Sigma$ je vstupní abeceda;
    \item $\delta$ je přechodová funkce (parciální funkce), \begin{compactitem}
        \item $\delta : Q \times \Sigma \rightarrow 2^{Q}$;
    \end{compactitem}
    \item $s$ je výchozí stav, \begin{compactitem}
        \item $s \in Q$;
    \end{compactitem}
    \item $F$ je množina koncových stavů, \begin{compactitem}
        \item $F \subseteq Q$;
    \end{compactitem}
\end{compactitem}

\paragraph*{Konfigurace} \todo{todo}

\paragraph*{Počáteční konfigurace} \todo{todo}

\paragraph*{Finální konfigurace} \todo{todo}

\paragraph*{Přechod} \todo{todo}

\paragraph*{Jazyk přijímaný KA} \todo{todo}

\paragraph*{Dosažitelné a nedosažitelné stavy} \todo{todo}

\paragraph*{Relace nerozlišitelnosti} \todo{todo}

\paragraph*{k-nerozlišitelnost} \todo{todo}

%%%%%%%%%%%%%%%%%%%%%%%%%%%%%%%%%%%%%%%%%%%%%%%%%%%%%%%%%%%%%%%%%%%%%%%%%%%%%%%%

\section{Varianty konečného automatu}

\paragraph*{Nedeterministický konečný automat} Nedeterministický konečný automat (NKA) je výchozí konečný automat (NKA = KA).

\paragraph*{Deterministický konečný automat} Deterministický konečný automat (DKA) se od NKA liší pouze tvarem přechodové funkce. Formálně: \begin{compactitem}
    \item $\delta$ je přechodová funkce (parciální funkce), \begin{compactitem}
        \item $\delta : Q \times \Sigma \rightarrow Q$
    \end{compactitem}
\end{compactitem}

\paragraph*{Rozšířený konečný automat} Rozšířený konečný automat (RKA) se od NKA liší pouze tvarem přechodové funkce. Rozšiřuje ji o tzv. epsilon přechody. Formálně: \begin{compactitem}
    \item $\delta$ je přechodová funkce (parciální funkce), \begin{compactitem}
        \item $\delta : Q \times \Sigma \cup \{ \epsilon \} \rightarrow 2^Q$
    \end{compactitem}
\end{compactitem}

\paragraph*{Úplný konečný automat} Úplný konečný automat se od NKA liší pouze tvarem přechodové funkce. Formálně: \begin{compactitem}
    \item $\delta$ je přechodová funkce (totální funkce), \begin{compactitem}
        \item $\delta : Q \times \Sigma \rightarrow 2^Q$
        \item $\forall q \in Q ~ \forall a \in \Sigma : \delta(q, a) \in Q$
    \end{compactitem}
\end{compactitem}

\paragraph*{Redukovaný úplný deterministický konečný automat} Úplný DKA nazýváme redukovaný, jestliže žádný $q \in Q$ není nedosažitelný a žádná dvojice $p, q \in Q$ není nerozlišitelná.

%%%%%%%%%%%%%%%%%%%%%%%%%%%%%%%%%%%%%%%%%%%%%%%%%%%%%%%%%%%%%%%%%%%%%%%%%%%%%%%%

\section{Minimalizace konečného automatu}

\paragraph*{Eliminace nedosažitelných stavů} \todo{todo}

\paragraph*{Převod na úplný DKA} \todo{todo}

\paragraph*{Převod na redukovaný úplný DKA} \todo{todo}

\paragraph*{Odstranění \textit{sink} stavu} \todo{todo}

%%%%%%%%%%%%%%%%%%%%%%%%%%%%%%%%%%%%%%%%%%%%%%%%%%%%%%%%%%%%%%%%%%%%%%%%%%%%%%%%

\section{Mihill-Nerodova věta}

\todo{todo}
