% VUT FIT MITAI
% MSZ 2021/2022
% Author: Vladimir Dusek
% Login: xdusek27

%%%%%%%%%%%%%%%%%%%%%%%%%%%%%%%%%%%%%%%%%%%%%%%%%%%%%%%%%%%%%%%%%%%%%%%%%%%%%%%%

% Path to figures
\graphicspath{{tin/chomskeho_hierarchie/figures}}

%%%%%%%%%%%%%%%%%%%%%%%%%%%%%%%%%%%%%%%%%%%%%%%%%%%%%%%%%%%%%%%%%%%%%%%%%%%%%%%%

\chapter{TIN~--~Regulární množiny, regulární výrazy a rovnice nad regulárními výrazy.}

%%%%%%%%%%%%%%%%%%%%%%%%%%%%%%%%%%%%%%%%%%%%%%%%%%%%%%%%%%%%%%%%%%%%%%%%%%%%%%%%

\section{Zdroje}

\begin{compactitem}
    \item \path{tin_2021_merged.pdf}
    \item \path{TIN_2020-09-29.mp4}
\end{compactitem}

%%%%%%%%%%%%%%%%%%%%%%%%%%%%%%%%%%%%%%%%%%%%%%%%%%%%%%%%%%%%%%%%%%%%%%%%%%%%%%%%

\section{Regulární množiny}

Regulární jazyky lze definovat pomocí regulárních gramatik, konečných automatů a regulárních množin.

\paragraph*{Definice} Nechť $\Sigma$ je konečná abeceda. Regulární množiny nad $\Sigma$ definujeme rekurzivně takto: \begin{compactitem}
    \item $\emptyset$ je regulární množina nad $\Sigma$;
    \item $\{ \epsilon \}$ je regulární množina nad $\Sigma$;
    \item $\{ a \}$ je regulární množina nad $\Sigma$ pro $\forall a \in \Sigma$;
    \item jsou-li $P$ a $Q$ regulární množiny nad $\Sigma$, pak také \begin{compactitem}
        \item $P \cup Q$,
        \item $P ~.~ Q$,
        \item $P^*$
    \end{compactitem}
    jsou regulární množiny nad $\Sigma$.
    \item Žádné jiné množiny, než ty, které lze získat pomocí výše uvedených pravidel, nejsou regulárními množinami.
\end{compactitem}

\paragraph*{Příklad (a)} Jazyk definovaný regulární množinou.

$$L_{RM} = ( \{ a \} \cup \{ d \} ) ~.~ ( \{ b \}^* ) ~.~ \{ c \}$$

%%%%%%%%%%%%%%%%%%%%%%%%%%%%%%%%%%%%%%%%%%%%%%%%%%%%%%%%%%%%%%%%%%%%%%%%%%%%%%%%

\section{Regulární výrazy}

Regulární výrazy jsou pouze zkrácený zápis regulárních množin.

\paragraph*{Definice} Regulární výrazy nad $\Sigma$ a regulární množiny, které označují, jsou rekurzivně definovány takto: \begin{compactitem}
    \item $\emptyset$ je regulární výraz označují regulární množinu $\emptyset$;
    \item $\epsilon$ je regulární výraz označují regulární množinu $\{ \epsilon \}$;
    \item $a$ je regulární výraz označují regulární množinu $\{ a \}$ pro $\forall a \in \Sigma$;
    \item jsou-li $p$ a $q$ regulární výrazy označují regulární množiny $P$ a $Q$, pak \begin{compactitem}
        \item $(p + q)$ je regulární výraz označující regulární množinu $P \cup Q$,
        \item $(pq)$ je regulární výraz označující regulární množinu $P ~.~ Q$,
        \item $(p^*)$ je regulární výraz označující regulární množinu $P^*$.
    \end{compactitem}
\end{compactitem}

\paragraph*{Příklad (b)} Jazyk definovaný regulárním výrazem, platí $L_{RM} = L_{RV}$. $$L_{RV} = (a + d) b^* c$$

\paragraph*{Příklad (c)} Jazyk definovaný regulární gramatikou $G$, platí $L_{RM} = L_{RV} = L(G)$.

$$ G = ( \{ S, A \}, \{ a, b, c, d \}, P, S ) $$

$$ P = \{ S \rightarrow aA ~|~ dA, A \rightarrow  bA ~|~ c \} $$

\paragraph*{Příklad (d)} Jazyk definovaný konečným automatem $M$, platí $L_{RM} = L_{RV} = L(G) = L(M)$.

$$ M = ( \{ q_0, q_1, q_2 \}, \{ a, b, c, d \}, \delta, q_0, \{ q_2 \} ) $$

$$ \delta(q_0, a) = q_1 ~,~ \delta(q_0, d) = q_1 ~,~ \delta(q_1, b) = q_1 ~,~ \delta(q_1, c) = q_2 $$

%%%%%%%%%%%%%%%%%%%%%%%%%%%%%%%%%%%%%%%%%%%%%%%%%%%%%%%%%%%%%%%%%%%%%%%%%%%%%%%%

\section{Rovnice nad regulárními výrazy}

\todo{todo}
