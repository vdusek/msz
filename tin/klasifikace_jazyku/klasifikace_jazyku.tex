% VUT FIT MITAI
% MSZ 2021/2022
% Author: Vladimir Dusek
% Login: xdusek27

%%%%%%%%%%%%%%%%%%%%%%%%%%%%%%%%%%%%%%%%%%%%%%%%%%%%%%%%%%%%%%%%%%%%%%%%%%%%%%%%

% Path to figures
\graphicspath{{tin/klasifikace_jazyku/figures}}

%%%%%%%%%%%%%%%%%%%%%%%%%%%%%%%%%%%%%%%%%%%%%%%%%%%%%%%%%%%%%%%%%%%%%%%%%%%%%%%%

\chapter{TIN~--~Klasifikace formálních jazyků (Chomského hierarchie), vlastnosti formálních jazyků a jejich rozhodnutelnost.}

% Todo:
%  - Normalni formy (Chomskeho, Greibachove), mozna az v ramci dalsich otazek

%%%%%%%%%%%%%%%%%%%%%%%%%%%%%%%%%%%%%%%%%%%%%%%%%%%%%%%%%%%%%%%%%%%%%%%%%%%%%%%%

\section{Zdroje}

\begin{compactitem}
    \item \path{tin_2021_merged.pdf}
    \item \path{TIN_2020-09-22.mp4}
    \item \path{TIN_2020-09-29.mp4}
    \item \path{TIN_2020-10-06.mp4}
    \item \path{TIN_2020-10-13.mp4}
\end{compactitem}

%%%%%%%%%%%%%%%%%%%%%%%%%%%%%%%%%%%%%%%%%%%%%%%%%%%%%%%%%%%%%%%%%%%%%%%%%%%%%%%%

\section{Úvod a kontext}

\begin{compactitem}
    \item Symbol, znak
    \item Abeceda~--~Konečná množina symbolů.
    \item Řetězec
    \item Konkatenace řetězců
    \item Reverze řetězce
    \item Podřetězec
    \item Délka řetězce
    \item Formální jazyk
    \item Doplněk jazyka
    \item Konkatenace jazyků
    \item Mocnina jazyka
    \item Iterace jazyka
\end{compactitem}

%%%%%%%%%%%%%%%%%%%%%%%%%%%%%%%%%%%%%%%%%%%%%%%%%%%%%%%%%%%%%%%%%%%%%%%%%%%%%%%%

\section{Gramatiky}

\paragraph*{Gramatika} Gramatika slouží k formální specifikaci jazyků, zejména pak nekonečných jazyků. Gramatika je čtveřice $G = (N, \Sigma, P, S)$, kde \begin{compactitem}
    \item $N$ je konečná množina neterminálů (pomocné syntaktické celky);
    \item $\Sigma$ je konečná množina terminálů (symbolů), \begin{compactitem}
        \item $N \cap \Sigma = \emptyset$;
    \end{compactitem}
    \item $P$ je konečná množina pravidel, \begin{compactitem}
        \item $P \subseteq (N \cup \Sigma)^* N (N \cup \Sigma)^* \times (N \cup \Sigma)^* $;
        \item prvek $(\alpha, \beta) \in P$ zapisujeme jako $\alpha \rightarrow \beta$;
    \end{compactitem}
    \item $S$ je výchozí neterminál, \begin{compactitem}
        \item $S \in N$.
    \end{compactitem}
\end{compactitem}

\paragraph*{Přímá derivace} Nechť $G = (N, \Sigma, P, S)$ je gramatika a nechť $\lambda, \mu \in (N \cup \Sigma)^*$. Mezi řetězci $\lambda$ a $\mu$ platí binární relace $\lambda \Rightarrow \mu$, nazývaná přímá derivace, pokud můžeme vyjádřit: $$\lambda = \gamma \alpha \delta$$ $$\mu = \gamma \beta \delta$$ kde $\gamma, \delta \in (N \cup \Sigma)^*$ a $\alpha \rightarrow \beta \in P$. Říkáme, že řetězec $\mu$ lze přímo derivovat z řetězce $\lambda$ v gramatice $G$.

\paragraph*{Derivace} Nechť $G = (N, \Sigma, P, S)$ je gramatika a nechť $\lambda, \mu \in (N \cup \Sigma)^*$. Mezi řetězci $\lambda$ a $\mu$ platí binární relace $\lambda \Rightarrow^+ \mu$, nazývaná derivace, jestliže existuje posloupnost přímých derivací $v_{i-1} \Rightarrow v_i,~~~ i=1, \dots, n,~~~ n \geq 1$, taková, že platí: $$\lambda = v_0 \Rightarrow v_1 \Rightarrow \dots \Rightarrow v_{n-1} \Rightarrow v_n = \mu$$. Tuto posloupnost nazýváme derivací délky $n$. Říkáme, že řetězec $\mu$ lze derivovat z řetězce $\lambda$ v gramatice $G$. Jestliže v gramatice $G$ platí pro řetězce $\lambda$ a $\mu$ relace $\lambda \Rightarrow^+ \mu$ a nebo identita $\lambda = \mu$, pak píšeme $\lambda \Rightarrow^* \mu$.

\paragraph*{Větná forma, věta, jazyk generovaný gramatikou} Nechť $G = (N, \Sigma, P, S)$ je gramatika. Řetězec $\alpha \in (N \cup \Sigma)^*$ nazýváme větnou formou, jestliže platí $S \Rightarrow^* \alpha$, tj. řetězec $\alpha$ je generovatelný z výchozího symbolu $S$. Větná forma, které obsahuje pouze terminální symboly, se nazývá věta. Jazyk $L(G)$ generovaný gramatikou $G$, je definován množinou všech vět $$L(G) = \{ w ~|~ w \in \Sigma^* \land S \Rightarrow^+ w \}$$.

%%%%%%%%%%%%%%%%%%%%%%%%%%%%%%%%%%%%%%%%%%%%%%%%%%%%%%%%%%%%%%%%%%%%%%%%%%%%%%%%

\section{Chomského hierarchie}

Chomského hierarchie je hierarchie tříd formálních gramatik generujících formální jazyky. Gramatiky jsou děleny dle tvaru jejich pravidel do 4 kategorií. Každé kategorii gramatik odpovídá kategorie formálních jazyků, které jsou gramatikami generovány~--~$\mathcal{L}_0, \mathcal{L}_1, \mathcal{L}_2, \mathcal{L}_3$. Platí: $2^{\Sigma^*} \supset \mathcal{L}_0 \supset \mathcal{L}_1 \supset \mathcal{L}_2 \supset \mathcal{L}_3$. Existují i jazyky mimo třídu $\mathcal{L}_0$, které neni možné gramatikami vyjádřit. Chomského hierarchie vymezuje popisnou a rozhodovací sílu dané třídy.

\begin{figure}[H]
    \centering
    \includegraphics[width=0.9\linewidth]{fj_hierarchy.pdf}
    \caption{Chomského hierarchie doplněna o další třídy a podtřídy jazyků. Jednotlivé třídy jsou vysvětleny dále. Teorii nerozhodnutelnosti je věnována jiná otázka.}
\end{figure}

% \textit{REL~--~Recursive Enumerable Languages}, \textit{RL~--~Recursive Languages}, \textit{DCFL~--~Deterministic Context Free Languages}.

\paragraph*{Třída jazyků $2^{\Sigma^*} - \mathcal{L}_0$} \begin{compactitem}
    \item Jazyky v této třídě není možné vyjádřit žádnou gramatikou.
    % \item Z hlediska teorie nerozhodnutelnosti, je možné problém transformovat (zakódovat) na jazyk. Problémy v této třídě nejsou ani částečně rozhodnutelné.
\end{compactitem}

\paragraph*{Typ 0 (obecné / neomezené / rekurzivně vyčíslitelné gramatiky)} \begin{compactitem}
    \item Pravidla v nejobecnějším tvaru: \begin{compactitem}
        \item $\alpha \rightarrow \beta,~~~ \alpha \in (N \cup \Sigma)^* N (N \cup \Sigma)^*,~~~ \beta \in (N \cup \Sigma)^*$
    \end{compactitem}

    \item Jazyky $\mathcal{L}_0$ jsou příjímány turingovými stroji (TS) s nekonečnou páskou. \begin{compactitem}
        \item Varianty: deterministický TS, nedeterministický TS a vícepáskový TS disponují stejnou vyjadřovací silou (jsou mezi sebou převoditelné).
    \end{compactitem}

    \item Mezi třídami jazyků $\mathcal{L}_0$ a $\mathcal{L}_1$ existuje třída tzv. rekurzivních jazyků ($\mathcal{L}_{Rec}$, \textit{Recursive Languages}). \begin{compactitem}
        \item Platí $\mathcal{L}_0 \supset \mathcal{L}_{Rec} \supset \mathcal{L}_1$.

        \item Rekurzivně vyčíslitelné jazyky: \begin{compactitem}
            \item Nechť $M$ je turingův stroj, který přijímá nějaký jazyk z $\mathcal{L}_0$. Pokud $M$ dostane na vstup $w \in L(M)$, tak pro něj zastaví a přijme ho. Avšak, pokud $M$ dostane na vstup $w \not\in L(M)$, tak buď zastaví a odmítne ho a nebo se zacyklí (nezastaví).
        \end{compactitem}

        \item Rekurzivní jazyky: \begin{compactitem}
            \item Nechť $M$ je turingův stroj, který přijímá nějaký jazyk z $\mathcal{L}_{Rec}$. Pokud $M$ dostane na vstup $w \in L(M)$, tak pro něj zastaví a přijme ho. Avšak, pokud $M$ dostane na vstup $w \not\in L(M)$, tak zastaví a odmítne ho. Turingův stroj, který zastaví pro jakýkoliv vstup se nazývá úplný turingův stroj.
        \end{compactitem}
    \end{compactitem}
\end{compactitem}

\paragraph*{Typ 1 (kontextové gramatiky)} \begin{compactitem}
    \item Pravidla ve tvaru: \begin{compactitem}
        \item $\alpha A \beta \rightarrow \alpha \gamma \beta,~~~ A \in N,~~~ \alpha, \beta \in (N \cup \Sigma)^*,~~~ \gamma \in (N \cup \Sigma)^+$
        \item A nebo $S \rightarrow \epsilon$, pokud se $S$ nevyskytuje na pravé straně žádného pravidla (neobsahují tzv. epsilon pravidla).
        \item Z tvaru pravidel vyplývá, že větné formy se v průběhu derivace nemohou zkracovat, ledaže se derivuje $S \Rightarrow \epsilon$, ale v takovém případě se S nemůže vyskytovat na pravé straně žádného pravidla.
        \item Formálně: pokud $\alpha \Rightarrow \beta$, pak $|\alpha| \leq |\beta|$ (s vyjímkou $S \Rightarrow \epsilon$).
    \end{compactitem}

    \item Jazyky $\mathcal{L}_1$ jsou příjímány nedeterministickým lineárně omezeným automatem (LOA). \begin{compactitem}
        \item Nedeterministický LOA je NTS s lineárně omezenou páskou v závislosti na délce vstupního řetězce.
        \item Není známo, zda deterministický LOA disponuje stejnou vyjadřovací sílou jako nedeterministický LOA.
    \end{compactitem}
\end{compactitem}

\paragraph*{Typ 2 (bezkontextové gramatiky)} \begin{compactitem}
    \item Pravidla ve tvaru: \begin{compactitem}
        \item $A \rightarrow \alpha,~~~ A \in N,~~~ \alpha \in (N \cup \Sigma)^*$
        \item Mohou obsahovat i tzv. epsilon pravidla (narozdíl od kontextových).
        \item Jak je to možné, když $\mathcal{L}_1 \supset \mathcal{L}_2$?
        \item Bezkontextové gramatiky s epsilon pravidly jsou převoditelné na kontextové gramatiky bez epsilon pravidel.
    \end{compactitem}

    \item Jazyky $\mathcal{L}_2$ jsou příjímány nedeterministickým zásobníkovým automatem (ZA). \begin{compactitem}
        \item Varianty: nedeterministický ZA a rozšířený nedeterministický ZA disponují stejnou vyjadřovací silou.
    \end{compactitem}

    \item Mezi třídami jazyků $\mathcal{L}_2$ a $\mathcal{L}_3$ existuje třída tzv. deterministických bezkontextových jazyků ($\mathcal{L}_{DCF}$, \textit{Deterministic Context Free Languages}). \begin{compactitem}
        \item Platí $\mathcal{L}_2 \supset \mathcal{L}_{DCF} \supset \mathcal{L}_3$.
        \item Jsou přijímány deterministickým ZA a deterministickým rozšířeným ZA. \todo{todo: ověřit}
    \end{compactitem}
\end{compactitem}

\paragraph*{Typ 3 (regulární gramatiky)} \begin{compactitem}
    \item Pravidla ve tvaru: \begin{compactitem}
        \item Pravolineární: $A \rightarrow wB~|~w,~~~ A,B \in N,~~~ w \in \Sigma^*$

        \item Levolineární: $A \rightarrow Bw~|~w,~~~ A,B \in N,~~~ w \in \Sigma^*$

        \item Pravoregulární: $A \rightarrow aB~|~a,~~~ A,B \in N,~~~ a \in \Sigma$

        \item Levoregulární: $A \rightarrow Ba~|~a,~~~ A,B \in N,~~~ a \in \Sigma$
    \end{compactitem}

    \item Gramatiky se všemi tvary pravidel jsou ekvivalentní.

    \item Jazyky $\mathcal{L}_3$ jsou příjímány nedeterministickým konečným automatem (KA). \begin{compactitem}
        \item Varianty: nedeterministický KA, deterministický KA a rozšířený KA disponují stejnou vyjadřovací silou.
    \end{compactitem}

    \item Třída konečných jazyků $\mathcal{L}_{Fin}$ je podmnožinou regulárních jazyků, platí $\mathcal{L}_{Fin} \subset \mathcal{L}_3$.
\end{compactitem}

%%%%%%%%%%%%%%%%%%%%%%%%%%%%%%%%%%%%%%%%%%%%%%%%%%%%%%%%%%%%%%%%%%%%%%%%%%%%%%%%

\section{Uzavřenost operací formálních jazyků}

\begin{compactitem}
    \item Které vlastnosti jazyků, resp. tříd jazyků, zkoumáme? \begin{compactitem}
        \item \textbf{Uzavřenost} vůči nějaké operaci~--~zda výsledek operace patří do stejné třidy.

        \item \textbf{Rozhodnutelnost} problému~--~zda je problém rozhodnutelný, částečně rozhodnutelný a nebo nerozhodnutelný.
    \end{compactitem}
\end{compactitem}

\paragraph*{Operace nad jazyky} \begin{compactitem}
    \item Základní množinové operace~--~sjednocení, průnik, doplněk.

    \item Základní jazykové operace~--~konkatenace, iterace, reverzace.

    \item Substituce~--~\todo{todo}.

    \item Morfismus~--~\todo{todo}.

    \item Inverzní morfismus~--~\todo{todo}
\end{compactitem}

\begin{figure}[H]
    \centering
    \includegraphics[width=1\linewidth]{uzavrenost_jazyku.png}
    \caption{Uzavřenost jednotlivých tříd jazyků vůči operacím.}
\end{figure}

\subsection{Uzávěrové vlastnosti regulárních jazyků}

\paragraph*{Sjednocení, průnik, doplněk, iterace, konkatenace} \begin{compactitem}
    \item Uzavřenost regulárních jazyků vůči operacím sjednocení, průnik, doplněk, iterace a konkatenace vyplývá z definice regulárních výrazů, resp. regulárních množin a ekvivalence regulárních množin a regulárních jazyků (\textit{viz otázka regulární výrazy}). \begin{compactitem}
        \item Alternativně např. pro sjednocení (pro ostatní operace je to analogické). Mějme regulární jazyky $L_1$ a $L_2$. Pak musí existovat KA $M_1$ a $M_2$, které je přijímají. Pokud jazyk $L_3 = L_1 \cup L_2$ je regulární, pak musí existovat KA $M_3 = M_1 \cup M_2$, který ho přijímá. % todo, toto neni presny
    \end{compactitem}
\end{compactitem}

\paragraph*{Doplněk ($co$)} Nechť $L \subseteq \Sigma^*$ je regulární jazyk. Nechť $M = (Q, \Sigma, \delta, q_0, F)$ je úplně definovaný KA, pro který platí $L = M(L)$. Pak KA $M' = (Q, \Sigma, \delta, q_0, Q - F)$ zřejmě přijímá jazyk $L_{co} = \Sigma^* - L$, tj. doplněk jazyka $L$.

\paragraph*{Průnik} Uzavřenost vzhledem k průniku plyne z de Morganových zákonů: $$L_1 \cap L_2 = \overline{\overline{L_1 \cap L_2}} = \overline{\overline{L_1} \cup \overline{L_2}}$$ a tedy $L_1, L_2 \in \mathcal{L}_3 \Leftrightarrow L_1 \cap L_2 \in \mathcal{L}_3$

\subsection{Uzávěrové vlastnosti bezkontextových jazyků}

\todo{todo}

\subsection{Uzávěrové vlastnosti kontextových jazyků}

\begin{figure}[H]
    \centering
    \includegraphics[width=1\linewidth]{uzaverove_vlastnosti_L1.pdf}
    \caption{Uzávěrové vlastnosti $\mathcal{L}_1$.}
\end{figure}

\subsection{Uzávěrové vlastnosti obecných jazyků}

\begin{compactitem}
    \item Zkoumáme pro rekurzivně vyčíslitelné $\mathcal{L}_{RE}$ a rekurzivní $\mathcal{L}_{Rec}$ zároveň.
\end{compactitem}

% \paragraph*{Sjednocení} Sjednocení dvou obecných jazyků je uzavřená operace. Oba obecné jazyky jsou přijímány nějakým TS $M_1$ a $M_2$. Sestrojíme nový NTS $M_{L_1 \cup L_2}$ tak, že sjednotíme po složkách stroje $M_1$ a $M_2$, zavedeme nový počáteční stav, z něj nedeterministické přechody přes $\Delta / \Delta$ do obou původních počátečních stavů a sloučíme původní koncové stavy do jediného nového koncového stavu.

% \paragraph*{Průnik} Průnik dvou obecných jazyků je uzavřená operace. Oba obecné jazyky jsou přijímány nějakým TS $M_1$ a $M_2$. Sestrojíme nový třípáskový NTS $M_{L_1 \cap L_2}$, okopíruje vstup z první pásky na druhou, na ní simuluje stroj $M_1$, pokud ten přijme, okopíruje vstup z první pásky na třetí, na ní simuluje stroj $M_2$, pokud i ten přijme, přijme i stroj $M_{L_1 \cap L_2}$.

\begin{figure}[H]
    \centering
    \includegraphics[width=1\linewidth]{uzaverove_vlasnosti_L0_01.pdf}
    \caption{Uzávěrové vlastnosti $\mathcal{L}_0$.}
\end{figure}

\begin{figure}[H]
    \centering
    \includegraphics[width=1\linewidth]{uzaverove_vlasnosti_L0_02.pdf}
    \caption{Uzávěrové vlastnosti $\mathcal{L}_0$.}
\end{figure}

\begin{figure}[H]
    \centering
    \includegraphics[width=1\linewidth]{uzaverove_vlasnosti_L0_03.pdf}
    \caption{Uzávěrové vlastnosti $\mathcal{L}_0$.}
\end{figure}

\begin{figure}[H]
    \centering
    \includegraphics[width=1\linewidth]{uzaverove_vlasnosti_L0_04.pdf}
    \caption{Uzávěrové vlastnosti $\mathcal{L}_0$.}
\end{figure}

\begin{figure}[H]
    \centering
    \includegraphics[width=1\linewidth]{uzaverove_vlasnosti_L0_05.pdf}
    \caption{Uzávěrové vlastnosti $\mathcal{L}_0$.}
\end{figure}

%%%%%%%%%%%%%%%%%%%%%%%%%%%%%%%%%%%%%%%%%%%%%%%%%%%%%%%%%%%%%%%%%%%%%%%%%%%%%%%%

\section{Rozhodnutelnost problémů formálních jazyků}

\paragraph*{Problémy nad jazyky} \begin{compactitem}
    \item Neprázdnost jazyka~--~Jazyk je neprázdný, pokud obsahuje aspoň 1 řetězec.

    \item Prázdnost jazyka~--~Jazyk je prázdný, pokud neobsahuje žádný řetězec.

    \item Konečnost jazyka~--~Jazyk je konečný, pokud obsahuje konečný počet řetězců.

    \item Náležitost řetězce~--~Řetězec náleží jazyku, pokud je jeho součástí.

    \item Univerzalita~--~Jazyk obsahuje všechny řetězce nad abecedou ($L = \Sigma^* $).

    \item Inkluze jazyků~--~Jazyk $L_1$ je inkluzí jazyka $L_2$, pokud je jazyk $L_1$ podmnožinou jazyka $L_2$.

    \item Ekvivalence gramatik~--~Gramatiky $G_1$ a $G_2$ jsou ekvivalentní, pokud platí, že\break $L(G_1) = L(G_2)$. $G_1$ a $G_2$ mohou být různé.
\end{compactitem}

\begin{figure}[H]
    \centering
    \includegraphics[width=1\linewidth]{rozhodnutelnost_jazyku.png}
    \caption{Rozhodnutelnost problémů v jednotlivých třídách jazyků.}
\end{figure}

\subsection{Rozhodnutelnost problémů regulárních jazyků}

\paragraph*{Regulárnost} \begin{compactitem}
    \item Nechť $L \subseteq \Sigma^*$, jak řešit problém: $L \in \mathcal{L}_3$? \begin{compactitem}

        \item Dokážeme existenci takového KA $M$, pro který platí $L(M) = L$.

        \item S využitím Myhill-Nerodovi věty. Najdeme relaci pravé kongruence s konečným indexem na $\Sigma^*$ takovou, že jazyk $L$ je sloučením některých tříd rozkladu.
    \end{compactitem}
\end{compactitem}

\paragraph*{Neregulárnost} \begin{compactitem}
    \item Nechť $L \subseteq \Sigma^*$, jak řešit problém: $L \not\in \mathcal{L}_3$? \begin{compactitem}

        \item S využitím Pumping Lemma pro regulární jazyky.

        \item S využitím Myhill-Nerodovi ukážeme, že neexistuje taková relace pravé kongruence s konečným indexem na $\Sigma^*$ taková, že jazyk $L$ je sloučením některých tříd rozkladu.
    \end{compactitem}
\end{compactitem}

\paragraph*{Prázdnost/neprázdnost} Problém je rozhodnutelný. Můžu sestrojit algoritmus, který pro $\forall w \in \Sigma^*$ zkusí, zda řetězec je přijímaný konečným automatem nebo není (problém dosažitelnosti konečného stavu z vychozího v grafu). Formálně: K jazyku $L \in \mathcal{L}_3$ sestrojíme úplně definovaný DKA $M$, takový, že $L = L(M)$, pak $$ L(M) \not= \emptyset \Leftrightarrow \exists q \in Q ~:~ (q \in F \land \text{q je dosažitelný z } q_0)$$

\paragraph*{Univerzalita} Problém je rozhodnutelný. K jazyku $L \in \mathcal{L}_3$ sestrojíme úplně definovaný DKA $M$, takový, že $L = L(M)$, pak $$ L(M) = \Sigma^* \Leftrightarrow \forall q \in Q ~:~ (q \in F ~\lor~ q \text{ je nedosažitelný z } q_0 )$$

\paragraph*{Náležitost} Problém je rozhodnutelný. K jazyku $L \in \mathcal{L}_3$ sestrojíme úplně definovaný DKA $M$, takový, že $L = L(M)$, pak $$ w \in L \Leftrightarrow (q_0, w) \vdash^* (q, \epsilon) \land q \in F $$

\paragraph*{Ekvivalence jazyků} Problém je rozhodnutelný. K jazyku $L_1, L_2 \in \mathcal{L}_3$ sestrojíme úplně definovaný DKA $M_1, M_2$, takový, že $L_1 = L(M_1)$ a $L_2 = L(M_2)$. Automaty $M_1, M_2$ minimalizujeme a porovnáme jejich struktury (ignorujeme identifikátory stavů).

\subsection{Rozhodnutelnost problémů bezkontextových jazyků}

\todo{todo}

\subsection{Rozhodnutelnost problémů kontextových jazyků}

\todo{todo}

\subsection{Rozhodnutelnost problémů obecných jazyků}

\todo{todo}



% - Jak ukazu, ze jazyk je bezkontextovy? Dokazu existenci takovyho ZA, ktery ho prijima.
% - Jak ukazu, ze jazyk neni bezkontextovy? PL

% - Jak ukazu, ze jazyk je rekurzivni? Dokazu existenci takovyho TS.
% - Jak ukazu, ze jazyk neni rekurzivni? Redukci z problemy, o kterym vim, ze neni rekurzivni (HP).
% - Redukce lze pouzit pro oba smery.
