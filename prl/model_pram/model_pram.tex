% VUT FIT MITAI
% MSZ 2021/2022
% Author: Vladimir Dusek
% Login: xdusek27

%%%%%%%%%%%%%%%%%%%%%%%%%%%%%%%%%%%%%%%%%%%%%%%%%%%%%%%%%%%%%%%%%%%%%%%%%%%%%%%%

% Path to figures
\graphicspath{{prl/model_pram/figures}}

%%%%%%%%%%%%%%%%%%%%%%%%%%%%%%%%%%%%%%%%%%%%%%%%%%%%%%%%%%%%%%%%%%%%%%%%%%%%%%%%

\chapter{PRL~--~Model PRAM, suma prefixů a její aplikace.}

%%%%%%%%%%%%%%%%%%%%%%%%%%%%%%%%%%%%%%%%%%%%%%%%%%%%%%%%%%%%%%%%%%%%%%%%%%%%%%%%

\section{Zdroje}

\begin{compactitem}
    \item \path{PRL_06_PRAM_MNG.pdf}
\end{compactitem}

%%%%%%%%%%%%%%%%%%%%%%%%%%%%%%%%%%%%%%%%%%%%%%%%%%%%%%%%%%%%%%%%%%%%%%%%%%%%%%%%

\section{Analýza algoritmů}

\begin{compactitem}
    \item Počet procesorů potřebných k řešení úlohy v závislosti na velikosti instance $n$.
    $$ p(n) $$

    \item Čas potřebný k vyřešení úlohy v krocích.
    $$ t(n) $$

    \item Cena paralelního řešení -- Jaké množství práce je potřebné na vyřešení problému.
    $$ c(n) = p(n) \cdot t(n) $$

    \item Optimální cena -- Taková cena, která je rovna optimálnímu sekvenčnímu algoritmu.
    $$ c(n) = t_{seq}(n) $$

    \item Zrychlení oproti sekvenčnímu algoritmu.
    $$ \displaystyle{ \frac{ t_{seq}(n) }{ t(n) } } $$

    \item Efektivnost.
    $$ \displaystyle{ \frac{ t_{seq}(n) }{ c(n) } } $$
\end{compactitem}

%%%%%%%%%%%%%%%%%%%%%%%%%%%%%%%%%%%%%%%%%%%%%%%%%%%%%%%%%%%%%%%%%%%%%%%%%%%%%%%%

\section{Dále}

\begin{compactitem}
    \item \todo{todo}
\end{compactitem}
