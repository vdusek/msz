% VUT FIT MITAI
% MSZ 2021/2022
% Author: Vladimir Dusek
% Login: xdusek27

%%%%%%%%%%%%%%%%%%%%%%%%%%%%%%%%%%%%%%%%%%%%%%%%%%%%%%%%%%%%%%%%%%%%%%%%%%%%%%%%

% Path to figures
\graphicspath{{prl/interakce_mezi_procesy/figures}}

%%%%%%%%%%%%%%%%%%%%%%%%%%%%%%%%%%%%%%%%%%%%%%%%%%%%%%%%%%%%%%%%%%%%%%%%%%%%%%%%

\chapter{PRL~--~Interakce mezi procesy a typické problémy paralelismu (synchronizační a komunikační mechanismy).}

%%%%%%%%%%%%%%%%%%%%%%%%%%%%%%%%%%%%%%%%%%%%%%%%%%%%%%%%%%%%%%%%%%%%%%%%%%%%%%%%

\section{Zdroje}

\begin{compactitem}
    \item \path{PRL_09_SynchSareMem_slidyA.pdf}
    \item \path{PRL_09_SynchSareMem_slidyB.pdf}
    \item \path{PRL_10_Modely_slidy.pdf}
\end{compactitem}

%%%%%%%%%%%%%%%%%%%%%%%%%%%%%%%%%%%%%%%%%%%%%%%%%%%%%%%%%%%%%%%%%%%%%%%%%%%%%%%%

\section{Úvod a kontext}

\begin{compactitem}
    \item Interakce mezi procesy \begin{compactitem}
        \item \textbf{Soupeření} (\textit{competition}) \begin{compactitem}
            \item Dva a více procesů se snaží v jeden okamžik přistoupit k nějakému sdílenému zdroji.
            \item Je třeba synchronizace, ke zdroji může přistoupit pouze jeden proces, ostatní musí čekat.
            \item Problémy: čtenáři a písaři, problém večeřících filosofů
        \end{compactitem}

        \item \textbf{Kooperace} (\textit{cooperation}) \begin{compactitem}
            \item Dva a více procesů na něčem spolupracují a nebo se na něčem musí dohodnout.
            \item Problémy: producenti a konzumenti
        \end{compactitem}
    \end{compactitem}

    \item \textbf{Kritická sekce} \begin{compactitem}
        \item Část programu, ve které se pracuje se sdíleným prostředkem.
        \item Je nutné zaručit, pokud procesy přistupují ke sdíleným zdrojům (místo ve sdílené paměti), musí být tento přístup výlučný.
    \end{compactitem}

    \item \textbf{Uváznutí} (\textit{deadlock}) \begin{compactitem}
        \item Uváznutí je situace, kdy proces(y) čekají na událost, která nemůže nastat.
        \item Příklad: nechť $p_1$ a $p_2$ jsou procesy a $r_1$ a $r_2$ sdílené prostředky. Proces $p_1$ disponuje $r_1$ a zároveň požaduje (čeká na) $r_2$. Proces $p_2$ disponuje $r_2$ a zároveň požaduje (čeká na) $r_1$.
        \item Může existovat uzavřená smyčka takovýchto procesů.
    \end{compactitem}

    \item \textbf{Vyhladovění} (\textit{starvation}) \begin{compactitem}
        \item Proces(y) se nemůže dostat ke sdílenému zdroji.
        \item Příklad: nechť $p_1$, $p_2$ a $p_3$ jsou procesy a $r$ je sdílené prostředek. Všechny procesy pracují s $r$. Procesy $p_1$ a $p_2$ se vzájemně střídají při práci s $r$, ale $p_3$ se k $r$ nikdy nedostane.
    \end{compactitem}

    \item \textbf{Souběh} (\textit{race condition}) \begin{compactitem}
        \item Procesy soupeří o sdílený zdroj, výsledek jejich operací je nepředvídatelný, jelikož může nastat nesprávné pořadí nebo načasování.
        \item Příklad: nechť $p_1$ a $p_2$ jsou procesy a $r$ je sdílený prostředek. Oba procesy chtějí hodnotu ve sdílené proměnné inkrementovat. Po zápisu obou procesů, může proměnná $r$ nabývat dvou různých hodnot $r_i \in \{ r_{i-1} +1 , r_{i-1} +2 \}$.
    \end{compactitem}

\end{compactitem}

%%%%%%%%%%%%%%%%%%%%%%%%%%%%%%%%%%%%%%%%%%%%%%%%%%%%%%%%%%%%%%%%%%%%%%%%%%%%%%%%

\section{Problémy}

\begin{compactitem}
    \item Typické problémy práce se sdílenou pamětí.
\end{compactitem}

\subsection{Producenti a konzumenti}

\begin{compactitem}
    \item \todo{todo}
\end{compactitem}

\noindent\begin{minipage}{\linewidth}
\begin{lstlisting}[language=c_language, caption={Naivní implementace s aktivním čekáním.}]
typedef int item;
extern int n;
item buffer[n - 1];
int in = 0, out = 0, counter = 0;

void Producent() {
    while (1) {
        // produce a new item
        item new_item = produce();
        // wait if the buffer is full
        while (counter == n)
            ;
        buffer[in] = new_item;
        in += 1 % n;
        counter += 1;
    }
}

void Consument() {
    while (1) {
        // wait until the buffer is not empty
        while (counter == 0)
            ;
        // produce a new item
        item consumend_item = buffer[out];
        out += 1 % n;
        counter -= 1;
        consume(consumend_item);
    }
}
\end{lstlisting}
\end{minipage}

\subsection{Čtenáři a písaři}

\begin{compactitem}
    \item \todo{todo}
\end{compactitem}

\subsection{Problém večeřících filosofů}

\begin{compactitem}
    \item \todo{todo}
\end{compactitem}

%%%%%%%%%%%%%%%%%%%%%%%%%%%%%%%%%%%%%%%%%%%%%%%%%%%%%%%%%%%%%%%%%%%%%%%%%%%%%%%%

\section{Softwarové řešení}

\begin{compactitem}
    \item Softwarové (algoritmické, pomocí zasílání zpráv) řešení práce se sdílenou pamětí.
\end{compactitem}

\subsection{Dekkerův algoritmus}

\begin{compactitem}
    \item \todo{todo}
\end{compactitem}

\subsection{Pettersonův algoritmus}

\begin{compactitem}
    \item \todo{todo}
\end{compactitem}

%%%%%%%%%%%%%%%%%%%%%%%%%%%%%%%%%%%%%%%%%%%%%%%%%%%%%%%%%%%%%%%%%%%%%%%%%%%%%%%%

\section{Hardwarové řešení}

\begin{compactitem}
    \item Hardwarové řešení práce se sdílenou pamětí.
\end{compactitem}

\subsection{Test and set}

\begin{compactitem}
    \item \todo{todo}
\end{compactitem}

\subsection{Swap}

\begin{compactitem}
    \item \todo{todo}
\end{compactitem}

\subsection{Bounded wait Test and set}

\begin{compactitem}
    \item \todo{todo}
\end{compactitem}

%%%%%%%%%%%%%%%%%%%%%%%%%%%%%%%%%%%%%%%%%%%%%%%%%%%%%%%%%%%%%%%%%%%%%%%%%%%%%%%%

\section{OS řešení}

\begin{compactitem}
    \item Řešení práce se sdílenou pamětí na úrovni operačního systému.
\end{compactitem}

\subsection{Semafory}

\begin{compactitem}
    \item \todo{todo}
\end{compactitem}

\subsection{Monitory}

\begin{compactitem}
    \item \todo{todo}
\end{compactitem}

\subsection{Kritické sekce}

\begin{compactitem}
    \item \todo{todo}
\end{compactitem}

%%%%%%%%%%%%%%%%%%%%%%%%%%%%%%%%%%%%%%%%%%%%%%%%%%%%%%%%%%%%%%%%%%%%%%%%%%%%%%%%

\section{Zasílání zpráv}

\begin{compactitem}
    \item Primitiva \begin{compactitem}
        \item send
        \item receive
    \end{compactitem}

    \item Druhy kanálů \begin{compactitem}
        \item Synchronní
        \item Asynchronní
    \end{compactitem}

    \item Modely komunikujících procesů \begin{compactitem}
        \item CSP (Communication Sequential Processes)
        \item Occam
        \item PI-Kalkul
        \item ADA
        \item Linda
    \end{compactitem}
\end{compactitem}
