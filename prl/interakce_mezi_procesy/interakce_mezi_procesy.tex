% VUT FIT MITAI
% MSZ 2021/2022
% Author: Vladimir Dusek
% Login: xdusek27

%%%%%%%%%%%%%%%%%%%%%%%%%%%%%%%%%%%%%%%%%%%%%%%%%%%%%%%%%%%%%%%%%%%%%%%%%%%%%%%%

% Path to figures
\graphicspath{{prl/interakce_mezi_procesy/figures}}

%%%%%%%%%%%%%%%%%%%%%%%%%%%%%%%%%%%%%%%%%%%%%%%%%%%%%%%%%%%%%%%%%%%%%%%%%%%%%%%%

\chapter{PRL~--~Interakce mezi procesy a typické problémy paralelismu (synchronizační a komunikační mechanismy).}

%%%%%%%%%%%%%%%%%%%%%%%%%%%%%%%%%%%%%%%%%%%%%%%%%%%%%%%%%%%%%%%%%%%%%%%%%%%%%%%%

\section{Zdroje}

\begin{compactitem}
    \item \path{PRL_09_SynchSareMem_slidyA.pdf}
    \item \path{PRL_09_SynchSareMem_slidyB.pdf}
    \item \path{PRL_10_Modely_slidy.pdf}
\end{compactitem}

%%%%%%%%%%%%%%%%%%%%%%%%%%%%%%%%%%%%%%%%%%%%%%%%%%%%%%%%%%%%%%%%%%%%%%%%%%%%%%%%

\section{Úvod a kontext}

\begin{compactitem}
    \item Interakce mezi procesy \begin{compactitem}
        \item \textbf{Soupeření} (\textit{competition}) \begin{compactitem}
            \item Dva a více procesů se snaží v jeden okamžik přistoupit k nějakému sdílenému zdroji.
            \item Je třeba synchronizace, ke zdroji může přistoupit pouze jeden proces, ostatní musí čekat.
            \item Problémy: čtenáři a písaři, problém večeřících filosofů
        \end{compactitem}

        \item \textbf{Kooperace} (\textit{cooperation}) \begin{compactitem}
            \item Dva a více procesů na něčem spolupracují a nebo se na něčem musí dohodnout.
            \item Problémy: producenti a konzumenti
        \end{compactitem}
    \end{compactitem}

    \item \textbf{Kritická sekce} \begin{compactitem}
        \item Část programu, ve které se pracuje se sdíleným prostředkem.
        \item Je nutné zaručit, pokud procesy přistupují ke sdíleným zdrojům (místo ve sdílené paměti), musí být tento přístup výlučný.
    \end{compactitem}

    \item \textbf{Uváznutí} (\textit{deadlock}) \begin{compactitem}
        \item Uváznutí je situace, kdy proces(y) čekají na událost, která nemůže nastat.
        \item Příklad: nechť $p_1$ a $p_2$ jsou procesy a $r_1$ a $r_2$ sdílené prostředky. Proces $p_1$ disponuje $r_1$ a zároveň požaduje (čeká na) $r_2$. Proces $p_2$ disponuje $r_2$ a zároveň požaduje (čeká na) $r_1$.
        \item Může existovat uzavřená smyčka takovýchto procesů.
    \end{compactitem}

    \item \textbf{Vyhladovění} (\textit{starvation}) \begin{compactitem}
        \item Proces(y) se nemůže dostat ke sdílenému zdroji.
        \item Příklad: nechť $p_1$, $p_2$ a $p_3$ jsou procesy a $r$ je sdílené prostředek. Všechny procesy pracují s $r$. Procesy $p_1$ a $p_2$ se vzájemně střídají při práci s $r$, ale $p_3$ se k $r$ nikdy nedostane.
    \end{compactitem}

    \item \textbf{Souběh} (\textit{race condition}) \begin{compactitem}
        \item Procesy soupeří o sdílený zdroj, výsledek jejich operací je nepředvídatelný, jelikož může nastat nesprávné pořadí nebo načasování.
        \item Příklad: nechť $p_1$ a $p_2$ jsou procesy a $r$ je sdílený prostředek. Oba procesy chtějí hodnotu ve sdílené proměnné inkrementovat. Po zápisu obou procesů, může proměnná $r$ nabývat dvou různých hodnot $r_i \in \{ r_{i-1} +1 , r_{i-1} +2 \}$.
    \end{compactitem}

\end{compactitem}

%%%%%%%%%%%%%%%%%%%%%%%%%%%%%%%%%%%%%%%%%%%%%%%%%%%%%%%%%%%%%%%%%%%%%%%%%%%%%%%%

\section{Problémy}

\begin{compactitem}
    \item Typické problémy práce se sdílenou pamětí.
\end{compactitem}

\subsection{Producenti a konzumenti}

\begin{compactitem}
    \item Mějme konečnou vyrovnávací paměť (buffer) a 2 procesy, které vykonávají jisté operace. \begin{compactitem}
        \item Producent, který vykonává operaci zápisu do bufferu (produkuje informace).
        \item Konzument, který vykonává operace čtení z bufferu (konzumuje informace).
    \end{compactitem}
    \item Operace posouvají ukazatele (\path{in}, \path{out}), které ukazují na místa, kde se bude číst, resp. zapisovat.
    \item Problémy: \begin{compactitem}
        \item Soubežná činnost producenta a konzumenta.
        \item Co se stane, když konzument nemá co konzumovat?
        \item Co se stane, když producent chce produkovat a buffer je plný?
    \end{compactitem}
\end{compactitem}

\subsubsection{Implementace s aktivním čekáním}

\noindent\begin{minipage}{\linewidth}
\begin{lstlisting}[language=c_language, caption={Implementace s aktivním čekáním.}]
int n = SIZE, in = 0, out = 0, counter = 0;
char buffer[n];

void producent() {
    while (1) {
        char item = produce(); // produce new item
        while (counter == n); // wait if the buffer is full
        buffer[in] = item;
        in = (in + 1) % n;
        counter += 1;
    }
}

void consument() {
    while (1) {
        while (counter == 0); // wait until the buffer is not empty
        char item = buffer[out]; // take item
        out = (out + 1) % n;
        counter -= 1;
        consume(item); // consume item
    }
}
\end{lstlisting}
\end{minipage}

\subsubsection{Implementace s využitím semaforů}

\begin{compactitem}
    \item Semafor S -- zajištění výhradního přístupu k bufferu.
    \item Semafor N -- určený pro synchronizování procesů na čísle $N = in - out$, což vyjadřuje počet prvků v bufferu.
\end{compactitem}

\noindent\begin{minipage}{\linewidth}
\begin{lstlisting}[language=c_language, caption={Implementace s využitím semaforů.}]
semaphore S, N;
int n = SIZE, in = 0, out = 0;
char buffer[n];
N.count = 0;
S.count = 1;

void producent() {
    while (1) {
        char item = produce(); // produce new item
        S.lock();
        buffer[in] = item;
        in += 1;
        S.unlock();
        N.unlock();
    }
}

void consument() {
    while (1) {
        N.lock();
        S.lock();
        char item = buffer[out]; // take item
        out += 1;
        S.unlock()
        consume(item); // consume item
    }
}
\end{lstlisting}
\end{minipage}

\subsubsection{Implementace s využitím semaforů a cyklického bufferu}

\begin{compactitem}
    \item Semafor S -- zajištění výhradního přístupu k bufferu.
    \item Semafor N -- určený pro synchronizování procesů na čísle $N = in - out$, což vyjadřuje počet prvků v bufferu.
    \item Semafor E -- určený pro synchronizování procesů na počtu volných míst v bufferu.
\end{compactitem}

\noindent\begin{minipage}{\linewidth}
\begin{lstlisting}[language=c_language, caption={Implementace s využitím semaforů a cyklického bufferu.}]
semaphore S, N, E;
int n = SIZE, in = 0, out = 0;
char buffer[n];
N.count = 0;
S.count = 1;
E.count = n;

void producent() {
    while (1) {
        char item = produce(); // produce new item
        E.lock();
        S.lock();
        buffer[in] = item;
        in = (in + 1) % n;
        S.unlock();
        N.unlock();
    }
}

void consument() {
    while (1) {
        N.lock();
        S.lock();
        char item = buffer[out]; // take item
        out = (out + 1) % n;
        S.unlock()
        E.unlock()
        consume(item); // consume item
    }
}
\end{lstlisting}
\end{minipage}

\subsection{Problém večeřících filosofů}

\begin{compactitem}
    \item Mějme 5 filozofů (procesů), kteří přemýšlejí (čekají) a jedí (pracují). Je k dispozici 5 vidliček a každý potřebuje 2 aby mohl jíst.
\end{compactitem}

\noindent\begin{minipage}{\linewidth}
\begin{lstlisting}[language=c_language, caption={Implementace problému večeřících filosofů s využitím semaforů.}]
semaphore T, Forks[5]; // number of resources

int n = SIZE, in = 0, out = 0;
char buffer[n];
T.count = 4; // number of process - 1

void philosopher(pid) { // pid in {0, 1, 2, 3, 4}
    while (1) {
        think();

        T.lock();
        Forks[pid].lock();
        Forks[(pid+1) % 5].lock();

        eat();

        Forks[(pid+1) % 5].unlock();
        Forks[pid].unlock();
        T.unlock();
    }
}
\end{lstlisting}
\end{minipage}

\subsection{Čtenáři a písaři}

\begin{compactitem}
    \item Více čtenářů potřebuje číst ze souboru a jeden, nebo více potřebuje zapisovat do souboru.
    \item Platí: \begin{compactitem}
        \item Současné čtení je povoleno libovolnému množství procesů.
        \item V daném okamžiku může zapisovat pouze jeden.
        \item Pokud někdo zapisuje tak současně nikdo nesmí číst.
    \end{compactitem}
\end{compactitem}

\noindent\begin{minipage}{\linewidth}
\begin{lstlisting}[language=c_language, caption={Implementace problému čtenářů a písařů s využitím semaforů.}]
semaphore M, W;
int reader_cnt = 0;
M.count = 1;
W.count = 1;

void reader() {
    while (1) {
        M.lock();
        reader_cnt += 1;
        if (reader_cnt == 0)
            W.lock();
        M.unlock();

        read()

        M.lock();
        reader_cnt -= 1;
        if (reader_cnt == 0)
            W.unlock();
        M.unlock();
    }
}

void writer() {
    while (1) {
        W.lock();
        write();
        W.unlock();
    }
}
\end{lstlisting}
\end{minipage}

%%%%%%%%%%%%%%%%%%%%%%%%%%%%%%%%%%%%%%%%%%%%%%%%%%%%%%%%%%%%%%%%%%%%%%%%%%%%%%%%

\section{Softwarové řešení}

\begin{compactitem}
    \item Softwarové (algoritmické, pomocí zasílání zpráv) řešení práce se sdílenou pamětí.
\end{compactitem}

\subsection{Dekkerův algoritmus}

\begin{compactitem}
    \item \todo{todo}
\end{compactitem}

\subsection{Pettersonův algoritmus}

\begin{compactitem}
    \item \todo{todo}
\end{compactitem}

%%%%%%%%%%%%%%%%%%%%%%%%%%%%%%%%%%%%%%%%%%%%%%%%%%%%%%%%%%%%%%%%%%%%%%%%%%%%%%%%

\section{Hardwarové řešení}

\begin{compactitem}
    \item Hardwarové řešení práce se sdílenou pamětí.
\end{compactitem}

\subsection{Test and set}

\begin{compactitem}
    \item \todo{todo}
\end{compactitem}

\subsection{Swap}

\begin{compactitem}
    \item \todo{todo}
\end{compactitem}

\subsection{Bounded wait Test and set}

\begin{compactitem}
    \item \todo{todo}
\end{compactitem}

%%%%%%%%%%%%%%%%%%%%%%%%%%%%%%%%%%%%%%%%%%%%%%%%%%%%%%%%%%%%%%%%%%%%%%%%%%%%%%%%

\section{OS řešení}

\begin{compactitem}
    \item Řešení práce se sdílenou pamětí na úrovni operačního systému.
\end{compactitem}

\subsection{Semafory}

\begin{compactitem}
    \item \todo{todo}
\end{compactitem}

\subsection{Monitory}

\begin{compactitem}
    \item \todo{todo}
\end{compactitem}

\subsection{Kritické sekce}

\begin{compactitem}
    \item \todo{todo}
\end{compactitem}
