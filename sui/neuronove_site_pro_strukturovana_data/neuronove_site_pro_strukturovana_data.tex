% VUT FIT MITAI
% MSZ 2021/2022
% Author: Vladimir Dusek
% Login: xdusek27

%%%%%%%%%%%%%%%%%%%%%%%%%%%%%%%%%%%%%%%%%%%%%%%%%%%%%%%%%%%%%%%%%%%%%%%%%%%%%%%%

% Path to figures
\graphicspath{{sui/neuronove_site_pro_strukturovana_data/figures}}

%%%%%%%%%%%%%%%%%%%%%%%%%%%%%%%%%%%%%%%%%%%%%%%%%%%%%%%%%%%%%%%%%%%%%%%%%%%%%%%%

\chapter{SUI~--~Neuronové sítě pro strukturovaná data (konvoluční a rekurentní sítě, motivace, základní vlastnosti, použití).}

%%%%%%%%%%%%%%%%%%%%%%%%%%%%%%%%%%%%%%%%%%%%%%%%%%%%%%%%%%%%%%%%%%%%%%%%%%%%%%%%

\section{Zdroje}

\begin{compactitem}
    \item \path{09-neural_networks.pdf}
    \item \path{10-sequences_and_language.pdf}
    \item \path{SUI_2019-12-02.mp4}
    \item \path{SUI_2019-12-09.mp4}
    \item \path{SUI_2019-12-16.mp4} % ?
\end{compactitem}

%%%%%%%%%%%%%%%%%%%%%%%%%%%%%%%%%%%%%%%%%%%%%%%%%%%%%%%%%%%%%%%%%%%%%%%%%%%%%%%%

\section{Strukturovaná data}

\begin{compactitem}
    \item Za strukturuovaná data považujeme obrázky, zvuk, text, \dots \begin{compactitem}
        \item Mají nějakou strukturu, nejsou to neuspořádaná data.
    \end{compactitem}

    \item Standardní vícevrstvé neuronové sítě by fungovali i pro tyto data, avšak lze využít vlastností strukturovaných dat, pro efektivnější práci neuronových sítí.

    O jaké vlastnosti jde: \begin{compactitem}
        \item Lokalita -- V případě obrázku, pixely, které se vyskytují blízko sebe, pravděpodobně patří stejnému objektu. Naopak pixely daleko od sebe spíše nepatří.

        \item Invariance zpracování vzhledem k pozici -- V případě obrázku, vychází myšlenky, že objekty v obraze se mohou pohybovat. Pokud se určitý objekt detekuje na jednom místě, je možné ho stejným způsobem detekovat i na jiném místě.
    \end{compactitem}
\end{compactitem}

%%%%%%%%%%%%%%%%%%%%%%%%%%%%%%%%%%%%%%%%%%%%%%%%%%%%%%%%%%%%%%%%%%%%%%%%%%%%%%%%

\section{Konvoluční neuronové sítě}

\todo{todo}

%%%%%%%%%%%%%%%%%%%%%%%%%%%%%%%%%%%%%%%%%%%%%%%%%%%%%%%%%%%%%%%%%%%%%%%%%%%%%%%%

\section{Rekurentní neuronové sítě}

\todo{todo}
