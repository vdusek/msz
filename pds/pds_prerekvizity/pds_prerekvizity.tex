% VUT FIT MITAI
% MSZ 2021/2022
% Author: Vladimir Dusek
% Login: xdusek27

%%%%%%%%%%%%%%%%%%%%%%%%%%%%%%%%%%%%%%%%%%%%%%%%%%%%%%%%%%%%%%%%%%%%%%%%%%%%%%%%

% Path to figures
\graphicspath{{pds/pds_prerekvizity/figures}}

%%%%%%%%%%%%%%%%%%%%%%%%%%%%%%%%%%%%%%%%%%%%%%%%%%%%%%%%%%%%%%%%%%%%%%%%%%%%%%%%

\chapter{PDS~--~Prerekvizity k ostatním otázkám.}

%%%%%%%%%%%%%%%%%%%%%%%%%%%%%%%%%%%%%%%%%%%%%%%%%%%%%%%%%%%%%%%%%%%%%%%%%%%%%%%%

% \section{Zdroje}

% \begin{compactitem}
%     \item \todo{todo}
% \end{compactitem}

%%%%%%%%%%%%%%%%%%%%%%%%%%%%%%%%%%%%%%%%%%%%%%%%%%%%%%%%%%%%%%%%%%%%%%%%%%%%%%%%

\paragraph*{ISO/OSI model} Referenční model ISO/OSI se používá jako názorný příklad řešení komunikace v počítačových sítích pomocí vrstevnatého modelu, kde jsou jednotlivé vrstvy nezávislé a snadno nahraditelné (alternativně TCP/IP model s L2, L3, L4 a L7). \begin{compactitem}

    \item \textbf{Aplikační vrstva} (L7, \textit{application layer}) \begin{compactitem}
        \item Zajišťuje zpracování dat na nejvyšší úrovni (reprezentace dat, kódování, řízení dialogu, \dots).
        \item Tvořena procesy a aplikacemi, které komunikují po síti.
        \item Bývá slučována s prezentační vrstvou (L6, prezentace dat a šifrování) a relační vrstvou (L5, koordinace a komunikace).
        \item Příklad protokolů: \begin{compactitem}
            \item Uživatelské~--~vykonávají služby přímo uživateli (Telnet, SSH, FTP, SMTP, HTTP, \dots)
            \item Systémové~--~zajišťují síťové funkce (DNS, DHCP, SNMP, BOOTP, \dots)
        \end{compactitem}
    \end{compactitem}

    \item \textbf{Transportní vrstva} (L4, \textit{transport layer}) \begin{compactitem}
        \item Rozděluje aplikační data (segmentace) na menší jednotky a zapouzdřuje je do segmentů (TCP) / datagramů (UDP).
        \item Vytváří logické spojení mezi procesy (přenáší data konkrétní aplikace ze zdrojového zařízení do aplikace na cílovém zařízení).
        \item Adresace: porty.
        \item Příklad protokolů: TCP, UDP, DCCP, SCTP, MP-TCP, QUIC
    \end{compactitem}

    \item \textbf{Síťová vrstva} (L3, \textit{network layer}) \begin{compactitem}
        \item Zapouzdřuje segmenty/datagramy do paketů.
        \item Řeší směrování.
        \item Adresace: IP adresa (logická adresa).
        \item Příklad protokolů: IPv4, IPv6, ARP, RARP, ICMP, IGMP
    \end{compactitem}

    \item \textbf{Linková vrstva} (L2, \textit{data link layer}, vrstva síťového rozhraní, \textit{network interface layer}) \begin{compactitem}
        \item Zapouzdřuje pakety do rámců.
        \item Zajišťuje \textit{hop-by-hop} doručení.
        \item Adresace: MAC adresa (fyzická adresace).
        \item Příklad protokolů: Ethernet, Token Ring, FDDI, X.25, Frame Relay
    \end{compactitem}

    \item \textbf{Fyzická vrstva} (L1, \textit{physical layer}) \begin{compactitem}
        \item Zajišťuje přenos bitů přes fyzické médium.
    \end{compactitem}
\end{compactitem}

\paragraph*{Adresace} \begin{compactitem}
    \item Port (transportní vrstva, L4) \begin{compactitem}
        \item Identifikuje aplikaci v rámci zařízení.
        \item Jak se mění při směrování paketu internetem: zůstává stejný s výjimkou překladu NAT. \begin{compactitem}
            \item Při komunikaci: soukromá síť $\rightarrow$ NAT $\rightarrow$ internet, se mění zdrojový port.
            \item Při komunikaci: internet $\rightarrow$ NAT $\rightarrow$ soukromá síť, se mění cílový port.
        \end{compactitem}
        \item Velikost: 16 bit
        \item Prostor: plochý
    \end{compactitem}
    \item IPv4, IPv6 (síťová vrstva, L3) \begin{compactitem}
        \item Identifikuje uzel v rámci sítě (tzv. logická adresace).
        \item Jak se mění při směrování paketu internetem: zůstává stejná s výjimkou překladu NAT. \begin{compactitem}
            \item Při komunikaci: soukromá síť $\rightarrow$ NAT $\rightarrow$ internet, se mění zdrojová IP.
            \item Při komunikaci: internet $\rightarrow$ NAT $\rightarrow$ soukromá síť, se mění cílová IP.
        \end{compactitem}
        \item Velikost: 32 bit, 128 bit
        \item Prostor: pseudohierarchie (A, B, C, D, E), pseudohierarchie (prefix + interface ID)
    \end{compactitem}
    \item MAC (linková vrstva, L2) \begin{compactitem}
        \item Identifikuje síťové rozhraní (síťovou kartu).
        \item Jak se mění při směrování paketu internetem: mění se \textit{hop-by-hop}.
        \item Velikost: 32 bit, 128 bit
        \item Prostor: ?
    \end{compactitem}
\end{compactitem}

\paragraph*{data, segment, datagram, paket, rámec, bit} \begin{compactitem}
    \item Data~--~aplikační vrstva (L7)
    \item Segment~--~transportní vrstva (L4), TCP
    \item Datagram~--~transportní vrstva (L4), UDP
    \item Paket~--~síťová vrstva (L3)
    \item Rámec~--~linková vrstva (L2)
    \item Bit~--~fyzická vrstva (L1)
\end{compactitem}

% src: https://linuxhint.com/network-osi-layers-explained
\begin{figure}[H]
    \centering
    \includegraphics[width=1\linewidth]{osi_model_linuxhint.pdf}
    \caption{Příklad OSI modelu z Linuxhint.}
\end{figure}

% src: https://cs.wikipedia.org/wiki/Soubor:OSI_Model_v1.svg
\begin{figure}[H]
    \centering
    \includegraphics[width=0.65\linewidth]{osi_model_wiki.pdf}
    \caption{Příklad OSI modelu z Wiki.}
\end{figure}

\begin{figure}[H]
    \centering
    \includegraphics[width=1\linewidth]{isoosi_tcpip.pdf}
    \caption{ISO/OSI vs TCP/IP.}
\end{figure}

\paragraph*{ACL} ACL (\textit{Access Control List}) je volitelná vrstva zabezpečení, která funguje jako brána firewall pro řízení provozu do jedné nebo více podsítí a z nich.

\paragraph*{NAT} NAT (\textit{Network Address Translation}) je metoda mapování IP adresního prostoru do jiného prostoru (typicky privátní adresy na veřejné adresy). Děje se tak úpravou hlaviček IP paketů během jejich přenosu přes směrovače (úprava zdrojové IP adresy a čísla portu). Směrovač si ukládá čtveřice $(\text{WAN\_IP}:\text{WAN\_port}, \text{LAN\_IP}:\text{LAN\_port})$ aby mohl prováděť i překlad zpět.

\begin{figure}[H]
    \centering
    \includegraphics[width=1\linewidth]{nat.pdf}
    \caption{Příklad překladu NAT.}
\end{figure}

\paragraph*{ARP} ARP (\textit{Address Resolution Protocol}) a RARP (\textit{Reverse ARP}) je protokol, který komunikuje na síťové vrstvě (L3) a zajišťuje \uv{překlad} IP adres na MAC adresy a obráceně. Pouze pro IPv4, pro IPv6 je pro stejný účel využíván protokol ICMPv6 a zpráva \textit{Neighbor Discovery}. Příklad využití: směrovač potřebuje získat MAC adresu next hopu (zná jeho IP adresu).

\paragraph*{ICMP} ICMP (\textit{Internet Control Message Protocol}) je protokol, který komunikuje na síťové vrstvě (L3) a slouží pro řízení toku a detekce nedosažitelných uzlů.

\paragraph*{MAC adresa} MAC adresa je fyzická adresa zařízení, resp. síťové karty (identifikátor na L2). Zařízení, a to jak koncová stanice, tak směrovač, mohou mít více síťových karet.

\paragraph*{IP adresa} IP adresa je logická adresa zařízení, resp. adresa síťové karty (identifikátor na L3). Přepínač nemá IP adresu vůbec, pracuje pouze na L2 vrstvě a paket nijak nemodifikuje. Typický směrovač má 2 IP adresy, jednu pro komunikaci v lokální síti (LAN) a druhou pro internet (WAN). Koncová stanice může mít rovněž více IP adres, např. připojení přes ethernet a wifi a nebo v případě telefonu, připojení přes mobilni data a wifi.

\paragraph*{Síťový tok} Síťový tok je posloupnost paketů (jednosměrná) identifikována čtveřicí (zdrojová IP, zdrojový port, cílová IP, cílový port).

\begin{figure}[H]
    \centering
    \includegraphics[width=1\linewidth]{operation_process_by_each_layer.png}
    \caption{Operace na jednotlivých vrstvách ISO modelu.}
\end{figure}
