% VUT FIT MITAI
% MSZ 2021/2022
% Author: Vladimir Dusek
% Login: xdusek27

%%%%%%%%%%%%%%%%%%%%%%%%%%%%%%%%%%%%%%%%%%%%%%%%%%%%%%%%%%%%%%%%%%%%%%%%%%%%%%%%

% Path to figures
\graphicspath{{pds/teorie_smerovani/figures}}

%%%%%%%%%%%%%%%%%%%%%%%%%%%%%%%%%%%%%%%%%%%%%%%%%%%%%%%%%%%%%%%%%%%%%%%%%%%%%%%%

\chapter{PDS -- Metody pro výpočet směrování v sítích (Bellman-Ford, Dijkstra, Path vector, DUAL).}

%%%%%%%%%%%%%%%%%%%%%%%%%%%%%%%%%%%%%%%%%%%%%%%%%%%%%%%%%%%%%%%%%%%%%%%%%%%%%%%%

\section{Metadata}

\begin{compactitem}
    \item Předmět: Přenos dat, počítačové sítě a protokoly (PDS)
    \item Přednáška:
    \begin{compactitem}
        \item \path{09-teorie-smerovani.pdf}
    \end{compactitem}
    \item Záznam:
    \begin{compactitem}
        \item 2021-04-16
    \end{compactitem}
\end{compactitem}

%%%%%%%%%%%%%%%%%%%%%%%%%%%%%%%%%%%%%%%%%%%%%%%%%%%%%%%%%%%%%%%%%%%%%%%%%%%%%%%%

\section{Úvod a kontext}

\textit{Viz. \uv{Úvod a kontext} v~předchozích otázkách z~tohoto předmětu.}

\paragraph*{Adresace} \begin{compactitem}
    \item Port \begin{compactitem}
        \item Identifikuje: aplikaci v rámci zařízení na transportní vrstvě (L4).
        \item Jak se mění při směrování paketu internetem: \todo{todo}
        \item Velikost: 16 bit
        \item Prostor: plochý
    \end{compactitem}
    \item IPv4, IPv6 \begin{compactitem}
        \item Identifikuje: uzel v rámci sítě na síťové vrstvě (L3).
        \item Jak se mění při směrování paketu internetem: měly zůstávat stejné, ale NAT toto porušuje
        \item Velikost: 32 bit, 128 bit
        \item Prostor: pseudohierarchie (A, B, C, D, E), pseudohierarchie (prefix + interface ID)
    \end{compactitem}
    \item MAC (\textit{Media Access Control}) \begin{compactitem}
        \item Identifikuje: síťové rozhraní (síťovou kartu) na linkové vrstvě (L2).
        \item Jak se mění při směrování paketu internetem: mění se \textit{hop-by-hop}
        \item Velikost: 32 bit, 128 bit
        \item Prostor: pseudohierarchie (A, B, C, D, E), pseudohierarchie (prefix + interface ID)
    \end{compactitem}
\end{compactitem}

\paragraph*{Datový tok} Identifikován $(src IP, src port, dst IP, dst port)$.

\paragraph*{Hierarchické směrování} Není možné směrovat přes celý internet. Vytváříme tzv. hierarchické autnomní systémy.

typy komunikace: broadcast, multicast, unicast

%%%%%%%%%%%%%%%%%%%%%%%%%%%%%%%%%%%%%%%%%%%%%%%%%%%%%%%%%%%%%%%%%%%%%%%%%%%%%%%%

\section{Distance Vector přístup}

\begin{compactitem}
    \item Decentralizované směrovací informace -- Každý uzel, zná pouze omezenou část topologie sítě. Konkrétně má informace pouze co sám zná a informace od svých sousedů.
    \item Používá se pro směrování uvnitř autonomních systémů.
    \item \textit{Single-metric}.
\end{compactitem}

\subsection*{Algoritmus Bellman-Ford}

% \begin{figure}[H]
%     \centering
%     \includegraphics[width=1\linewidth]{}
%     \caption{Základní činnost směrovače.}
% \end{figure}

\todo{todo}

\subsection*{Algoritmus EIGRP}

Enhanced Interior Gateway Routing Protocol

Algoritmus DUAL (diffusing update algorithm) je algoritmus používaný směrovacím protokolem EIGRP společnosti Cisco, který zajišťuje, že daná trasa je přepočítána globálně, kdykoli by mohla způsobit směrovací smyčku.

\todo{todo}

%%%%%%%%%%%%%%%%%%%%%%%%%%%%%%%%%%%%%%%%%%%%%%%%%%%%%%%%%%%%%%%%%%%%%%%%%%%%%%%%

\section{Link State přístup}

\begin{compactitem}
    \item Globální směrovací informace -- Každý uzel zná celou topologii sítě. Na začátku si uzly vymění informace o topologii sítě.
    \item Používá se pro směrování uvnitř autonomních systémů.
    \item \textit{Single-metric}.
\end{compactitem}

\subsection*{Dijkstrův algoritmus}

\todo{todo}

%%%%%%%%%%%%%%%%%%%%%%%%%%%%%%%%%%%%%%%%%%%%%%%%%%%%%%%%%%%%%%%%%%%%%%%%%%%%%%%%

\section{Path Vector přístup}

\begin{compactitem}
    \item Globální směrovací informace.
    \item Na síť je pohlíženo jako na množinu autonomních systémů.
    \item Používá se pro směrování mezi autonomními systémy.
    \item \textit{Multi-metric}.
\end{compactitem}

\subsection*{Path Vector algoritmus}

\todo{todo}
