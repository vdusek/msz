% VUT FIT MITAI
% MSZ 2021/2022
% Author: Vladimir Dusek
% Login: xdusek27

%%%%%%%%%%%%%%%%%%%%%%%%%%%%%%%%%%%%%%%%%%%%%%%%%%%%%%%%%%%%%%%%%%%%%%%%%%%%%%%%

% Path to figures
\graphicspath{{pds/smerovace/figures}}

%%%%%%%%%%%%%%%%%%%%%%%%%%%%%%%%%%%%%%%%%%%%%%%%%%%%%%%%%%%%%%%%%%%%%%%%%%%%%%%%

\chapter{PDS -- Základní funkce směrovače, zpracování paketů ve směrovači, typy přepínání a architektur.}

%%%%%%%%%%%%%%%%%%%%%%%%%%%%%%%%%%%%%%%%%%%%%%%%%%%%%%%%%%%%%%%%%%%%%%%%%%%%%%%%

\section{Metadata}

\begin{compactitem}
    \item Předmět: Přenos dat, počítačové sítě a protokoly (PDS)
    \item Přednáška:
    \begin{compactitem}
        \item \path{05-routing.pdf}
    \end{compactitem}
    \item Záznam:
    \begin{compactitem}
        \item 2021-03-12
    \end{compactitem}
\end{compactitem}

%%%%%%%%%%%%%%%%%%%%%%%%%%%%%%%%%%%%%%%%%%%%%%%%%%%%%%%%%%%%%%%%%%%%%%%%%%%%%%%%

\section{Úvod a kontext}

\paragraph*{NAT} Network Address Translation \todo{todo}

\paragraph*{ACL} Access Control List \todo{todo}

\paragraph*{Maximum Transmission Unit (MTU)} Největší velikost paketu, kterou lze v síti odeslat (přes výstupní rozhraní síťového prvku).

\paragraph*{Fragmentace paketů} Uzel v síti (směrovač) dostane paket o velikosti $n$. Paket má přeposlat přes výstupní rozhraní do sítě, ve které je $MTU < n$. Aby paket mohl být odeslán, musí být rozdělen (fragmentován) na více menších paketů (fragmenty) a odeslán po částech. Na straně příjemce pak musí nastat opačný proces~--~defragmentace.

%%%%%%%%%%%%%%%%%%%%%%%%%%%%%%%%%%%%%%%%%%%%%%%%%%%%%%%%%%%%%%%%%%%%%%%%%%%%%%%%

\section{Obecná architektura směrovače}

Směrovač je síťové zařízení, které předává datové pakety mezi počítačovými sítěmi. \begin{compactitem}
    \item Směrovač pracuje s pakety (síťová vrstva, L3).
\end{compactitem}

\paragraph*{Funkce směrovače} \begin{compactitem}
    \item Tvoří komunikace mezi lokálními sítěmi.
    \item Provádí překlad NAT.
    \item Klasifikuje a filtruje pakety (firewall).
    \item Fragmentace a defragmentace.
\end{compactitem}

\paragraph*{Činnost směrovače} \begin{compactitem}
    \item Vypouzdření paketu z L2 (odebrání L2 hravičky) a kontrola jestli je v pořádku (pomocí kontrolního součtu).
    \item Vyhledání cesty kam se má paket směrovat a překlad adresy NAT (pomocí směrovací tabulky).
    \item Určení cílové MAC adresy na základě cílové IP adresy (pošle ARP dotaz).
    \item Určení výstupního rozhraní.
    \item Sestavení výsledného paketu podle výstupního rozhraní (zapouzdření do příslušné L2 technologie -- přidání L2 hlavičky).
    \item \todo{todo} ACL
\end{compactitem}

\paragraph*{Co ovlivňuje propustnost směrovače} \begin{compactitem}
    \item Rozbalení paketu.
    \item Vyhledání směrovací cesty.
    \item Překlad NAT.
    \item Vyhledání cílové MAC adresy.
    \item Zabalení paketu do správné technologie.
\end{compactitem}
